\documentclass[a4j,12pt]{jarticle}

\usepackage{otf}
\usepackage{graphicx,color}
\usepackage{amsmath,amssymb}
\usepackage{hyperref}
%\usepackage{emath}
\usepackage{emath,emathE,emathMw}
%\usepackage{xcolor}
\usepackage{bm,enumerate}
\usepackage{pict2e}
\usepackage{ketpic2e,ketlayer2e}

\setmargin{20}{20}{20}{20}

\begin{document}

\begin{center}
{\bf \large MNR法の使い方}
\end{center}
\hfill \today

MNR法は,三角形について2底角の半角の正接$m,n$と内接円の半径$r$で諸量を$m,n,r$の有理式で表す方法である.%和算に現れる問題を解く方法として相当に有効である.
すなわち,$m=\tan\frac{B}{2},n=\tan\frac{C}{2}$%
および内接円の半径を$r$とおく.

\begin{center}
%%% /Users/takatoosetsuo/Dropbox/2023ketpic/sibauraronbun/paper/fig/putT.tex 
%%% Generator=putT.cdy 
{\unitlength=8mm%
\begin{picture}%
(7.49,6.43)(-4.24,-2.71)%
\linethickness{0.008in}%%
\polyline(1.8,0)(1.786,0.226)(1.743,0.448)(1.674,0.663)(1.577,0.867)(1.456,1.058)%
(1.312,1.232)(1.147,1.387)(0.964,1.52)(0.766,1.629)(0.556,1.712)(0.337,1.768)(0.113,1.796)%
(-0.113,1.796)(-0.337,1.768)(-0.556,1.712)(-0.766,1.629)(-0.964,1.52)(-1.147,1.387)%
(-1.312,1.232)(-1.456,1.058)(-1.577,0.867)(-1.674,0.663)(-1.743,0.448)(-1.786,0.226)%
(-1.8,0)(-1.786,-0.226)(-1.743,-0.448)(-1.674,-0.663)(-1.577,-0.867)(-1.456,-1.058)%
(-1.312,-1.232)(-1.147,-1.387)(-0.964,-1.52)(-0.766,-1.629)(-0.556,-1.712)(-0.337,-1.768)%
(-0.113,-1.796)(0.113,-1.796)(0.337,-1.768)(0.556,-1.712)(0.766,-1.629)(0.964,-1.52)%
(1.147,-1.387)(1.312,-1.232)(1.456,-1.058)(1.577,-0.867)(1.674,-0.663)(1.743,-0.448)%
(1.786,-0.226)(1.8,0)%
%
\polyline(1.8,0)(1.786,0.226)(1.743,0.448)(1.674,0.663)(1.577,0.867)(1.456,1.058)%
(1.312,1.232)(1.147,1.387)(0.964,1.52)(0.766,1.629)(0.556,1.712)(0.337,1.768)(0.113,1.796)%
(-0.113,1.796)(-0.337,1.768)(-0.556,1.712)(-0.766,1.629)(-0.964,1.52)(-1.147,1.387)%
(-1.312,1.232)(-1.456,1.058)(-1.577,0.867)(-1.674,0.663)(-1.743,0.448)(-1.786,0.226)%
(-1.8,0)(-1.786,-0.226)(-1.743,-0.448)(-1.674,-0.663)(-1.577,-0.867)(-1.456,-1.058)%
(-1.312,-1.232)(-1.147,-1.387)(-0.964,-1.52)(-0.766,-1.629)(-0.556,-1.712)(-0.337,-1.768)%
(-0.113,-1.796)(0.113,-1.796)(0.337,-1.768)(0.556,-1.712)(0.766,-1.629)(0.964,-1.52)%
(1.147,-1.387)(1.312,-1.232)(1.456,-1.058)(1.577,-0.867)(1.674,-0.663)(1.743,-0.448)%
(1.786,-0.226)(1.8,0)%
%
\polyline(0.403,3.28)(-3.86,-1.8)(2.881,-1.8)(0.403,3.28)%
%
\polyline(0,0)(0.403,3.28)%
%
\polyline(0,0)(-3.86,-1.8)%
%
\polyline(0,0)(2.881,-1.8)%
%
\polyline(-3.26,-1.8)(-3.265,-1.725)(-3.279,-1.651)(-3.302,-1.579)(-3.334,-1.511)%
(-3.375,-1.447)(-3.423,-1.389)(-3.475,-1.341)%
%
\settowidth{\Width}{$(m)$}\setlength{\Width}{-0.5\Width}%
\settoheight{\Height}{$(m)$}\settodepth{\Depth}{$(m)$}\setlength{\Height}{-0.5\Height}\setlength{\Depth}{0.5\Depth}\addtolength{\Height}{\Depth}%
\put( -3.140, -1.470){\hspace*{\Width}\raisebox{\Height}{$(m)$}}%
%
\polyline(-2.61,-1.8)(-2.62,-1.643)(-2.649,-1.489)(-2.698,-1.34)(-2.729,-1.273)%
%
\settowidth{\Width}{$\frac{B}{2}$}\setlength{\Width}{-0.5\Width}%
\settoheight{\Height}{$\frac{B}{2}$}\settodepth{\Depth}{$\frac{B}{2}$}\setlength{\Height}{-0.5\Height}\setlength{\Depth}{0.5\Depth}\addtolength{\Height}{\Depth}%
\put( -2.400, -1.480){\hspace*{\Width}\raisebox{\Height}{$\frac{B}{2}$}}%
%
\polyline(2.618,-1.261)(2.559,-1.293)(2.498,-1.338)(2.443,-1.389)(2.395,-1.447)(2.355,-1.511)%
(2.323,-1.579)(2.299,-1.651)(2.285,-1.725)(2.281,-1.8)%
%
\settowidth{\Width}{$(n)$}\setlength{\Width}{-0.5\Width}%
\settoheight{\Height}{$(n)$}\settodepth{\Depth}{$(n)$}\setlength{\Height}{-0.5\Height}\setlength{\Depth}{0.5\Depth}\addtolength{\Height}{\Depth}%
\put(  2.210, -1.380){\hspace*{\Width}\raisebox{\Height}{$(n)$}}%
%
\polyline(1.823,-1.139)(1.785,-1.198)(1.718,-1.34)(1.67,-1.489)(1.64,-1.643)(1.631,-1.8)%
%
\settowidth{\Width}{$\frac{C}{2}$}\setlength{\Width}{-0.5\Width}%
\settoheight{\Height}{$\frac{C}{2}$}\settodepth{\Depth}{$\frac{C}{2}$}\setlength{\Height}{-0.5\Height}\setlength{\Depth}{0.5\Depth}\addtolength{\Height}{\Depth}%
\put(  1.510, -1.410){\hspace*{\Width}\raisebox{\Height}{$\frac{C}{2}$}}%
%
\settowidth{\Width}{A}\setlength{\Width}{-0.5\Width}%
\settoheight{\Height}{A}\settodepth{\Depth}{A}\setlength{\Height}{\Depth}%
\put(  0.400,  3.405){\hspace*{\Width}\raisebox{\Height}{A}}%
%
\settowidth{\Width}{B}\setlength{\Width}{-1\Width}%
\settoheight{\Height}{B}\settodepth{\Depth}{B}\setlength{\Height}{-0.5\Height}\setlength{\Depth}{0.5\Depth}\addtolength{\Height}{\Depth}%
\put( -3.985, -1.800){\hspace*{\Width}\raisebox{\Height}{B}}%
%
\settowidth{\Width}{C}\setlength{\Width}{0\Width}%
\settoheight{\Height}{C}\settodepth{\Depth}{C}\setlength{\Height}{-0.5\Height}\setlength{\Depth}{0.5\Depth}\addtolength{\Height}{\Depth}%
\put(  3.005, -1.800){\hspace*{\Width}\raisebox{\Height}{C}}%
%
\polyline(0,-1.8)(0.006,-1.788)(0.012,-1.776)(0.018,-1.763)(0.024,-1.751)(0.03,-1.739)%
(0.036,-1.726)(0.041,-1.714)(0.047,-1.701)(0.052,-1.689)(0.058,-1.676)(0.063,-1.663)%
(0.068,-1.651)(0.074,-1.638)(0.079,-1.625)(0.084,-1.613)(0.088,-1.6)(0.093,-1.587)%
(0.098,-1.574)(0.102,-1.561)(0.107,-1.548)(0.111,-1.535)(0.116,-1.522)(0.12,-1.509)%
(0.124,-1.496)(0.128,-1.483)(0.132,-1.47)(0.136,-1.457)(0.14,-1.444)(0.144,-1.431)%
(0.147,-1.418)(0.151,-1.404)(0.154,-1.391)(0.158,-1.378)(0.161,-1.365)(0.164,-1.351)%
(0.167,-1.338)(0.17,-1.325)(0.173,-1.311)(0.176,-1.298)(0.178,-1.284)(0.181,-1.271)%
(0.184,-1.258)(0.186,-1.244)(0.188,-1.231)(0.19,-1.217)(0.193,-1.204)(0.195,-1.19)%
(0.197,-1.176)(0.198,-1.163)(0.2,-1.149)%
%
\polyline(0.2,-0.651)(0.198,-0.637)(0.197,-0.624)(0.195,-0.61)(0.193,-0.596)(0.19,-0.583)%
(0.188,-0.569)(0.186,-0.556)(0.184,-0.542)(0.181,-0.529)(0.178,-0.516)(0.176,-0.502)%
(0.173,-0.489)(0.17,-0.475)(0.167,-0.462)(0.164,-0.449)(0.161,-0.435)(0.158,-0.422)%
(0.154,-0.409)(0.151,-0.396)(0.147,-0.382)(0.144,-0.369)(0.14,-0.356)(0.136,-0.343)%
(0.132,-0.33)(0.128,-0.317)(0.124,-0.304)(0.12,-0.291)(0.116,-0.278)(0.111,-0.265)%
(0.107,-0.252)(0.102,-0.239)(0.098,-0.226)(0.093,-0.213)(0.088,-0.2)(0.084,-0.187)%
(0.079,-0.175)(0.074,-0.162)(0.068,-0.149)(0.063,-0.137)(0.058,-0.124)(0.052,-0.111)%
(0.047,-0.099)(0.041,-0.086)(0.036,-0.074)(0.03,-0.061)(0.024,-0.049)(0.018,-0.037)%
(0.012,-0.024)(0.006,-0.012)(0,0)%
%
\settowidth{\Width}{$r$}\setlength{\Width}{-0.5\Width}%
\settoheight{\Height}{$r$}\settodepth{\Depth}{$r$}\setlength{\Height}{-0.5\Height}\setlength{\Depth}{0.5\Depth}\addtolength{\Height}{\Depth}%
\put(  0.220, -0.900){\hspace*{\Width}\raisebox{\Height}{$r$}}%
%
\polyline(-3.86,-1.8)(-3.828,-1.813)(-3.796,-1.826)(-3.764,-1.839)(-3.732,-1.851)%
(-3.699,-1.864)(-3.667,-1.876)(-3.634,-1.888)(-3.602,-1.899)(-3.569,-1.911)(-3.536,-1.922)%
(-3.503,-1.933)(-3.47,-1.944)(-3.437,-1.954)(-3.404,-1.965)(-3.371,-1.975)(-3.338,-1.984)%
(-3.305,-1.994)(-3.271,-2.003)(-3.238,-2.013)(-3.205,-2.021)(-3.171,-2.03)(-3.138,-2.039)%
(-3.104,-2.047)(-3.07,-2.055)(-3.036,-2.063)(-3.003,-2.07)(-2.969,-2.077)(-2.935,-2.084)%
(-2.901,-2.091)(-2.867,-2.098)(-2.833,-2.104)(-2.799,-2.11)(-2.765,-2.116)(-2.731,-2.122)%
(-2.696,-2.127)(-2.662,-2.132)(-2.628,-2.137)(-2.594,-2.142)(-2.559,-2.146)(-2.525,-2.151)%
(-2.49,-2.155)(-2.456,-2.158)(-2.422,-2.162)(-2.387,-2.165)(-2.353,-2.168)(-2.318,-2.171)%
(-2.284,-2.174)(-2.249,-2.176)(-2.215,-2.178)(-2.18,-2.18)%
%
\polyline(-1.68,-2.18)(-1.646,-2.178)(-1.611,-2.176)(-1.576,-2.174)(-1.542,-2.171)%
(-1.507,-2.168)(-1.473,-2.165)(-1.438,-2.162)(-1.404,-2.158)(-1.37,-2.155)(-1.335,-2.151)%
(-1.301,-2.146)(-1.267,-2.142)(-1.232,-2.137)(-1.198,-2.132)(-1.164,-2.127)(-1.13,-2.122)%
(-1.095,-2.116)(-1.061,-2.11)(-1.027,-2.104)(-0.993,-2.098)(-0.959,-2.091)(-0.925,-2.084)%
(-0.891,-2.077)(-0.857,-2.07)(-0.824,-2.063)(-0.79,-2.055)(-0.756,-2.047)(-0.723,-2.039)%
(-0.689,-2.03)(-0.656,-2.021)(-0.622,-2.013)(-0.589,-2.003)(-0.555,-1.994)(-0.522,-1.984)%
(-0.489,-1.975)(-0.456,-1.965)(-0.423,-1.954)(-0.39,-1.944)(-0.357,-1.933)(-0.324,-1.922)%
(-0.291,-1.911)(-0.258,-1.899)(-0.226,-1.888)(-0.193,-1.876)(-0.161,-1.864)(-0.129,-1.851)%
(-0.096,-1.839)(-0.064,-1.826)(-0.032,-1.813)(0,-1.8)%
%
\settowidth{\Width}{$\frac{r}{m}$}\setlength{\Width}{-0.5\Width}%
\settoheight{\Height}{$\frac{r}{m}$}\settodepth{\Depth}{$\frac{r}{m}$}\setlength{\Height}{-0.5\Height}\setlength{\Depth}{0.5\Depth}\addtolength{\Height}{\Depth}%
\put( -1.930, -2.190){\hspace*{\Width}\raisebox{\Height}{$\frac{r}{m}$}}%
%
\polyline(0,-1.8)(0.023,-1.809)(0.045,-1.819)(0.068,-1.828)(0.091,-1.837)(0.114,-1.845)%
(0.137,-1.854)(0.16,-1.862)(0.183,-1.871)(0.206,-1.879)(0.23,-1.887)(0.253,-1.895)%
(0.276,-1.903)(0.3,-1.91)(0.323,-1.918)(0.347,-1.925)(0.37,-1.932)(0.394,-1.939)(0.417,-1.946)%
(0.441,-1.952)(0.465,-1.959)(0.488,-1.965)(0.512,-1.971)(0.536,-1.977)(0.56,-1.983)%
(0.584,-1.989)(0.608,-1.994)(0.632,-2)(0.656,-2.005)(0.68,-2.01)(0.704,-2.015)(0.728,-2.02)%
(0.752,-2.024)(0.776,-2.029)(0.8,-2.033)(0.825,-2.037)(0.849,-2.041)(0.873,-2.045)%
(0.897,-2.048)(0.922,-2.052)(0.946,-2.055)(0.97,-2.058)(0.995,-2.061)(1.019,-2.064)%
(1.044,-2.067)(1.068,-2.07)(1.093,-2.072)(1.117,-2.074)(1.141,-2.076)(1.166,-2.078)%
(1.19,-2.08)%
%
\polyline(1.69,-2.08)(1.715,-2.078)(1.739,-2.076)(1.764,-2.074)(1.788,-2.072)(1.813,-2.07)%
(1.837,-2.067)(1.861,-2.064)(1.886,-2.061)(1.91,-2.058)(1.935,-2.055)(1.959,-2.052)%
(1.983,-2.048)(2.007,-2.045)(2.032,-2.041)(2.056,-2.037)(2.08,-2.033)(2.104,-2.029)%
(2.129,-2.024)(2.153,-2.02)(2.177,-2.015)(2.201,-2.01)(2.225,-2.005)(2.249,-2)(2.273,-1.994)%
(2.297,-1.989)(2.321,-1.983)(2.345,-1.977)(2.368,-1.971)(2.392,-1.965)(2.416,-1.959)%
(2.44,-1.952)(2.463,-1.946)(2.487,-1.939)(2.51,-1.932)(2.534,-1.925)(2.557,-1.918)%
(2.581,-1.91)(2.604,-1.903)(2.628,-1.895)(2.651,-1.887)(2.674,-1.879)(2.697,-1.871)%
(2.72,-1.862)(2.743,-1.854)(2.766,-1.845)(2.789,-1.837)(2.812,-1.828)(2.835,-1.819)%
(2.858,-1.809)(2.881,-1.8)%
%
\settowidth{\Width}{$\frac{r}{n}$}\setlength{\Width}{-0.5\Width}%
\settoheight{\Height}{$\frac{r}{n}$}\settodepth{\Depth}{$\frac{r}{n}$}\setlength{\Height}{-0.5\Height}\setlength{\Depth}{0.5\Depth}\addtolength{\Height}{\Depth}%
\put(  1.440, -2.090){\hspace*{\Width}\raisebox{\Height}{$\frac{r}{n}$}}%
%
\polyline(-4.24,0)(3.248,0)%
%
\polyline(0,-2.707)(0,3.72)%
%
\settowidth{\Width}{$x$}\setlength{\Width}{0\Width}%
\settoheight{\Height}{$x$}\settodepth{\Depth}{$x$}\setlength{\Height}{-0.5\Height}\setlength{\Depth}{0.5\Depth}\addtolength{\Height}{\Depth}%
\put(  3.312,  0.000){\hspace*{\Width}\raisebox{\Height}{$x$}}%
%
\settowidth{\Width}{$y$}\setlength{\Width}{-0.5\Width}%
\settoheight{\Height}{$y$}\settodepth{\Depth}{$y$}\setlength{\Height}{\Depth}%
\put(  0.000,  3.783){\hspace*{\Width}\raisebox{\Height}{$y$}}%
%
\settowidth{\Width}{I}\setlength{\Width}{-1\Width}%
\settoheight{\Height}{I}\settodepth{\Depth}{I}\setlength{\Height}{\Depth}%
\put( -0.125,  0.125){\hspace*{\Width}\raisebox{\Height}{I}}%
%
\end{picture}}%
\end{center}

このとき,$\mathrm{B}(-\frac{r}{m},\ -r),\ \mathrm{C}(\frac{r}{n},\ -r)$となる.
頂角$A$については

$$\tan\frac{A}{2}=\tan\frac{\pi-B-C}{2}=\cot\frac{B+C}{2}=\dfrac{1-mn}{m+n}$$

\noindent
となる.また,頂点Aの座標も直線AB,\ ACの交点としてやはり$m,\ n$の有理式で求められる.
$$\left(\frac{r(n-m)}{1-mn},\frac{1+mn}{1-mn}\right)$$
なお,通常は底辺を下側にとるが,その場合は$1-mn>0$となる.\vspace{1mm}

辺BCの長さは$\frac{r}{m}+\frac{r}{n}$\vspace{1mm}であり,他の辺も同様に計算される.
 $$\mathrm{AB}=\frac{r(1 + m^2)}{(m(1 - mn)},\ \mathrm{AC}=\frac{r(1 + n^2)}{(n(1 - mn)}$$
%\vspace{1mm}

\noindent
外心,外接円,重心,垂心,傍心,傍接円,三角形の面積なども$m,\ n$の有理式で表される.

MNR法では半角の正接が重要となる.
そこで,$\alpha$\ ($-\pi<\alpha<\pi$)について,
$\tan\frac{\alpha}{2}=t$
となる$\alpha$を$(t)$と表すことにする.
すなわち,$\alpha=2\tan^{-1}t$である.
例えば,
$$(1)=\dfrac{\pi}{2},\ (\sqrt{3})=\dfrac{2\pi}{3},\ (\sqrt{2})=$$

\section{MNR法のライブラリ}

\subsection{3つの基本関数と大域変数}

\noindent
\hspace*{2zw}\Ltab{10zw}{putT(m,n,r)}角$B,\ C$がそれぞれ$(m),\ (n)$で内心が原点の三角形をおく.\\
\hspace*{2zw}\Ltab{10zw}{slideT(pt1,pt2)}pt1がpt2に一致するように平行移動する.\\
\hspace*{2zw}\Ltab{10zw}{rotateT(m,pt)}ptを中心に(m)だけ回転する.\vspace{2mm}

これらを実行すると,頂点,辺の長さ,5心などが次の大域変数に代入または変更される.\vspace{2mm}\\
\hspace*{2zw}\Ltab{8zw}{頂点}vtxT,\ vtxL,\ vtxR\\
\hspace*{2zw}\Ltab{8zw}{辺}edgB,\ edgL,\ edgR\\
\hspace*{2zw}\Ltab{8zw}{内心,\ 外心}inC,\ inR,\ cirC,\ cirR\\
\hspace*{2zw}\Ltab{8zw}{垂心,\ 重心}ortC,\ barC\\
\hspace*{2zw}\Ltab{8zw}{傍心,\ 傍接円}exCa,exRa,exCb,exRb,exCc,exRc\\
\hspace*{2zw}\Ltab{8zw}{面積Sとs}area,\ halfPer ($s=\frac{a+b+c}{2})$\vspace{2mm}


これらのうち,頂点以外はputTだけによって決定される.
\subsection{汎用変数}
$\alpha=(t)$の補角$\pi-\alpha$は
$\tan\frac{\pi-\alpha}{2}=\cot\frac{\alpha}{2}=\frac{1}{t}$\vspace{1mm}より
$(\frac{1}{t})$と表される.同様に.加法定理によって,余角$\frac{\pi}{2}-\alpha$は
$\tan(\frac{\pi}{4}-\frac{\alpha}{2})=\frac{1-t}{1+t}$\vspace{1mm}となることなどから,
次の関数を定義する.\\
\hspace*{2zw}\Ltab{8zw}{補角}supA(t) (:=1/t)\\
\hspace*{2zw}\Ltab{8zw}{余角}comA(t) (:=(1-t)/(1+t)\\
\hspace*{2zw}\Ltab{8zw}{角の和}plusA(t1,t2) (:=(t1+t2)/(1-t1*t2))\\
\hspace*{2zw}\Ltab{8zw}{角の差}minusA(t1,t2) (:=(t1-t2)/(1+t1*t2))\\

\noindent
それ以外にも,以下のような汎用的な関数が定義されている.\vspace{2mm}\\
\hspace*{2zw}\Ltab{12zw}{頂角}angT(m,n)\\
\hspace*{2zw}\Ltab{12zw}{numer(f)}方程式(=0)の分子を因数分解 :=factor(num(ratsimp(f)))\\
\hspace*{2zw}\Ltab{12zw}{frev(eq,rep)}eqにrepを代入して分数式を簡単化\\
\hspace*{2zw}\Ltab{12zw}{frevL(eqL,rep)}リストeqLにrepを代入して分数式を簡単化\\
\hspace*{2zw}\Ltab{12zw}{frfactor(eq,rep)}eqにrepを代入して分数式を因数分解して簡単化)\\
\hspace*{2zw}\Ltab{12zw}{nthfactor(pol,k)}多項式のk番目の因子を返す(望む結果にならない場合も)\\
\hspace*{2zw}\Ltab{12zw}{dotProd(v1,v2)}内積\\
\hspace*{2zw}\Ltab{12zw}{crossProd(v1,v2)}外積\\
\hspace*{2zw}\Ltab{12zw}{lenSeg2(p1,[p2])}p1 [p2-p1]の長さの平方\\
\hspace*{2zw}\Ltab{12zw}{meetLine(pts1,pts2)}2線分の交点 (ptsは2点のリスト)\\
\hspace*{2zw}\Ltab{12zw}{edge(A,B)}辺AB(frfactorで簡単化)\\
\hspace*{2zw}\Ltab{12zw}{edg2m(c,a,b)}三角形ABCにおいて,頂点Cのmの値\\
\hspace*{2zw}\Ltab{12zw}{cos2m(c)}cosの値が$c$である角のmの値

\section{Maximaのコマンドと関数}

\begin{itemize}
\item 代入はコロン($=$ではない) (ex) A:vtxT\vspace{-2mm}
\item リストは[ ]で囲む.(ex) eqL:[s1,s2];  eqL[1] (=s1)\vspace{-2mm}
\item  方程式を解く\\
 単独の方程式 solve(eq,x) (注)方程式に=はつけない\\
 連立方程式 algsys(eqL,[x,y]) (solveでも解けることもある)\\
 解は sol:[x=a1, x=a2],\ [[x=a, y=b]など\\
 解を代入するには,frevやfrevLを用いる\\
  eq:x-a; sol:solve(eq,x); x:frev(sol)\vspace{-2mm}
\item  partfrec 部分分数分解  \vspace{-2mm}
\end{itemize}

\section{Cindyのスクリプト例}

\begin{verbatim}
 cmdL1=concat(Mxbatch("mnr"),[ //mnrライブラリを読み込む
   //以前のスクリプト,例えば,cmdL1に追加する場合は,concat(cmdL1,[
 "putT(m,n,r); slideT(cirC,[0,0])",
 "aA:angT(m,n)", //頂角 (これが定数であることを示す)
 "eq1:edgB-a; eq2:cirR-R",
 "sol:solve([eq1,eq2],[n,r])",
 "v:frevL([vtxT,vtxL,vtxR,n,aA],sol[2])",
 "A:v[1]; B:v[2]; C:v[3]; n:v[4]; aA:v[5]",
 "end"
]);
var1="sol::A::B::C::n::aA"; //値が返される変数文字列(リストに変換される)
if(contains(Ch,1), //Ch=[1]の場合(画面のボタンで選択)
 Nchoice(1,0..4);Setfiles(Namecdy+"1"); // 画面に進行のボタンをおいた場合
 CalcbyMset(var1,"mxans1",cmdL1,[""]); // Maximaを実行
    //var1の各変数に結果の数式(文字列)が代入される
 R=3; a=4; m=tanhalf(80); //仮の値
 v=Parsev("A::B::C"); //A,B,Cを評価して,リストにする
 Listplot("1",v_[1,2,3,1]); //三角形を描く
 Circledata("1",[[0,0],R]); //外接円を描く
);
\end{verbatim}
  
\end{document}

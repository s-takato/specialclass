%%% title presen24202
\documentclass[landscape,10pt]{ujarticle}
\usepackage{pict2e}
\usepackage{ketpic2e,ketlayer2e}
\special{papersize=\the\paperwidth,\the\paperheight}
\usepackage{ketslide}
\usepackage{amsmath,amssymb}
\usepackage{bm,enumerate}
\usepackage[dvipdfmx]{graphicx}
\usepackage{color}
\usepackage[dvipdfmx,colorlinks=true,linkcolor=blue,filecolor=blue]{hyperref}
\usepackage{emath,emathE,emathMw}

\def\deg#1{#1^{\circ}}
\newcommand{\monthday}{0626}

\definecolor{slidecolora}{cmyk}{0.98,0.13,0,0.43}
\definecolor{slidecolorb}{cmyk}{0.2,0,0,0}
\definecolor{slidecolorc}{cmyk}{0.2,0,0,0}
\definecolor{slidecolord}{cmyk}{0.2,0,0,0}
\definecolor{slidecolore}{cmyk}{0,0,0,0.5}
\definecolor{slidecolorf}{cmyk}{0,0,0,0.5}
\definecolor{slidecolori}{cmyk}{0.98,0.13,0,0.43}
\def\setthin#1{\def\thin{#1}}
\setthin{0}
\newcounter{pagectr}
\setcounter{pagectr}{1}
\newcommand{\slidepage}[1][\monthday-]{%
\setcounter{ketpicctra}{18}%

\begin{layer}{118}{0}
\putnotew{130}{-\theketpicctra.05}{\small#1\thepage/\pageref{pageend}}
\end{layer}

}

\setmargin{25}{145}{15}{100}

\ketslideinit

\pagestyle{empty}

\begin{document}

\begin{layer}{120}{0}
\putnotese{0}{0}{{\Large\bf
\color[cmyk]{1,1,0,0}

\begin{layer}{120}{0}
{\Huge \putnotes{60}{20}{インストール状況}}
\putnotes{60}{50}{高遠節夫}
\end{layer}

}
}
\end{layer}

\def\mainslidetitley{22}
\def\ketcletter{slidecolora}
\def\ketcbox{slidecolorb}
\def\ketdbox{slidecolorc}
\def\ketcframe{slidecolord}
\def\ketcshadow{slidecolore}
\def\ketdshadow{slidecolorf}
\def\slidetitlex{6}
\def\slidetitlesize{1.3}
\def\mketcletter{slidecolori}
\def\mketcbox{yellow}
\def\mketdbox{yellow}
\def\mketcframe{yellow}
\def\mslidetitlex{62}
\def\mslidetitlesize{2}

\color{black}
\Large\bf\boldmath
\addtocounter{page}{-1}

\renewcommand{\eda}[2][\theedactr]{%
\Ltab{\theedawidth mm}{[#1]\ #2}%
\addtocounter{edactr}{1}%
}
\renewcommand{\slidepage}[1][s]{%
\setcounter{ketpicctra}{18}%
\if#1m \setcounter{ketpicctra}{1}\fi
\hypersetup{linkcolor=black}%
\begin{layer}{118}{0}
\putnotee{115}{-\theketpicctra.05}{\small\monthday-\thepage/\pageref{pageend}}
\end{layer}\hypersetup{linkcolor=blue}
}
%%%%%%%%%%%%

%%%%%%%%%%%%%%%%%%%%

\mainslide{ 復習(微分係数・導関数)}


\slidepage[m]
%%%%%%%%%%%%%

%%%%%%%%%%%%%%%%%%%%

\newslide{定義(質問)}

\vspace*{18mm}

\slidepage
\textinit

\begin{layer}{120}{0}
\addtext{8}{\ten}{微分係数$f'(a)$は定点$a$における接線の傾き}
\addtext{12}{}{$f'(a)=\dlim_{z \to a}\hakom{\dfrac{f(z)-f(a)}{z-a}}$}
\addtext[8]{8}{問\monbannoadd}{$f'(a)$の定義式をかけ}
\addban
\addtext[4]{8}{\ten}{導関数$f'(x)$は$a$を変数と考え,$x$とおいたもの}
\addtext{12}{}{$f'(x)=\dlim_{z \to x}\hakom{\dfrac{f(z)-f(a)}{z-a}}$}
\addtext[8]{8}{問\monbannoadd}{$f'(x)$の定義式をかけ}
\addban
\end{layer}

%%%%%%%%%%%%%

%%%%%%%%%%%%%%%%%%%%


\newslide{導関数の書き方}

\vspace*{18mm}

\slidepage
\textinit

\begin{layer}{120}{0}
\addtext{8}{\ten}{導関数$f'(x)$を求めることを「微分する」}
\addtext{8}{\ten}{関数$y=f(x)$を変数$x$で微分する}
\addtext{16}{}{$y',\ f'(x)$(ラグランジュ)}
\addtext{16}{}{$\bunsuu{dy}{dx}$(ライプニッツ)}
\addtext[4]{8}{[例]}{$y=f(x)=x^2$}
\addtext{16}{}{$y'=f'(x)=f'=\bigl(x^2\bigr)'=2x$}
\addtext[4]{16}{}{$\bunsuu{dy}{dx}=\bunsuu{df}{dx}=\bunsuu{d}{dx}f(x)=\bunsuu{d}{dx}(x^2)=2x$}
\end{layer}

%%%%%%%%%%%%

%%%%%%%%%%%%%%%%%%%%


\mainslide{ いろいろな関数の微分}


\slidepage[m]
%%%%%%%%%%%%%

%%%%%%%%%%%%%%%%%%%%

\newslide{$c,ax,ax^2$の微分}

\vspace*{18mm}


\begin{layer}{120}{0}
\putnotew{96}{73}{\hyperlink{para0pg1}{\fbox{\Ctab{2.5mm}{\scalebox{1}{\scriptsize $\mathstrut||\!\lhd$}}}}}
\putnotew{101}{73}{\hyperlink{para1pg1}{\fbox{\Ctab{2.5mm}{\scalebox{1}{\scriptsize $\mathstrut|\!\lhd$}}}}}
\putnotew{108}{73}{\hyperlink{para1pg4}{\fbox{\Ctab{4.5mm}{\scalebox{1}{\scriptsize $\mathstrut\!\!\lhd\!\!$}}}}}
\putnotew{115}{73}{\hyperlink{para1pg5}{\fbox{\Ctab{4.5mm}{\scalebox{1}{\scriptsize $\mathstrut\!\rhd\!$}}}}}
\putnotew{120}{73}{\hyperlink{para1pg5}{\fbox{\Ctab{2.5mm}{\scalebox{1}{\scriptsize $\mathstrut \!\rhd\!\!|$}}}}}
\putnotew{125}{73}{\hyperlink{para2pg1}{\fbox{\Ctab{2.5mm}{\scalebox{1}{\scriptsize $\mathstrut \!\rhd\!\!||$}}}}}
\putnotee{126}{73}{\scriptsize\color{blue} 5/5}
\end{layer}

\slidepage
\textinit

\begin{layer}{120}{0}
\addtext{8}{\ten}{$(c)'=\dlim_{z\to x}\bunsuu{c-c}{z-x}=0$}
\addtext[4]{8}{\ten$(ax)'$}{$=\dlim_{z\to x}\bunsuu{az-ax}{z-x}=\dlim_{z\to x}\bunsuu{a(z-x)}{z-x}=a$}
\addtext[8]{8}{\ten$(ax^2)'$}{$=\dlim_{z\to x}\bunsuu{az^2-ax^2}{z-x}
=\dlim_{z\to x}\bunsuu{a(z^2-x^2)}{z-x}$}
\addtext[8]{8}{\phantom{\ten$(ax^2)'$}}{$=\dlim_{z\to x}\bunsuu{a(z-x)(z+x)}{z-x}$}
\addtext[8]{8}{\phantom{\ten$(ax^2)'$}}{$=\dlim_{z\to x}a(z+x)=2ax$}
\end{layer}


\newslide{$x^3$の微分}

\vspace*{18mm}

\slidepage
\begin{itemize}
\item
$(x^3)'=\dlim_{z \to x}\bunsuu{z^3-x^3}{z-x}$\hfill(1)  \vspace{-2mm}%%
\item
次の因数分解公式を用いる\\
\hspace*{2zw}$z^3-x^3=(z-x)(z^2+zx+x^2)$
\item
(1)$=\dlim_{z \to x}\bunsuu{(z-x)(z^2+zx+x^2)}{z-x}$\\
\phantom{(1)}$=\dlim_{z \to x}(z^2+zx+x^2)=3x^2$\\
\hspace*{4zw}\fbox{$(x^3)'=3x^2$}
\item
[問]\monban$(x^4)'$を求めよ
\end{itemize}

\newslide{微分の公式}

\vspace*{18mm}


\begin{layer}{120}{0}
\putnotew{96}{73}{\hyperlink{para1pg6}{\fbox{\Ctab{2.5mm}{\scalebox{1}{\scriptsize $\mathstrut||\!\lhd$}}}}}
\putnotew{101}{73}{\hyperlink{para2pg1}{\fbox{\Ctab{2.5mm}{\scalebox{1}{\scriptsize $\mathstrut|\!\lhd$}}}}}
\putnotew{108}{73}{\hyperlink{para2pg5}{\fbox{\Ctab{4.5mm}{\scalebox{1}{\scriptsize $\mathstrut\!\!\lhd\!\!$}}}}}
\putnotew{115}{73}{\hyperlink{para2pg6}{\fbox{\Ctab{4.5mm}{\scalebox{1}{\scriptsize $\mathstrut\!\rhd\!$}}}}}
\putnotew{120}{73}{\hyperlink{para2pg6}{\fbox{\Ctab{2.5mm}{\scalebox{1}{\scriptsize $\mathstrut \!\rhd\!\!|$}}}}}
\putnotew{125}{73}{\hyperlink{para3pg1}{\fbox{\Ctab{2.5mm}{\scalebox{1}{\scriptsize $\mathstrut \!\rhd\!\!||$}}}}}
\putnotee{126}{73}{\scriptsize\color{blue} 6/6}
\end{layer}

\slidepage

\begin{layer}{120}{0}
\putnotese{72}{5}{\scalebox{0.9}{%%% /Users/takatoosetsuo/Dropbox/2021polytech/108/fig/bibun4.tex 
%%% Generator=doukansuu2.cdy 
{\unitlength=7mm%
\begin{picture}%
(8,8)(-4,-4)%
\special{pn 8}%
%
\special{pn 12}%
\special{pa -1102  -413}\special{pa  1102  -413}%
\special{fp}%
\special{pn 8}%
\special{pa    20  -413}\special{pa   -20  -413}%
\special{fp}%
\settowidth{\Width}{$c$}\setlength{\Width}{-1\Width}%
\settoheight{\Height}{$c$}\settodepth{\Depth}{$c$}\setlength{\Height}{-\Height}%
\put(-0.0714286,1.4285714){\hspace*{\Width}\raisebox{\Height}{$c$}}%
%
\special{pa -1102    -0}\special{pa  1102    -0}%
\special{fp}%
\special{pa     0  1102}\special{pa     0 -1102}%
\special{fp}%
\settowidth{\Width}{$x$}\setlength{\Width}{0\Width}%
\settoheight{\Height}{$x$}\settodepth{\Depth}{$x$}\setlength{\Height}{-0.5\Height}\setlength{\Depth}{0.5\Depth}\addtolength{\Height}{\Depth}%
\put(4.0714286,0.0000000){\hspace*{\Width}\raisebox{\Height}{$x$}}%
%
\settowidth{\Width}{$y$}\setlength{\Width}{-0.5\Width}%
\settoheight{\Height}{$y$}\settodepth{\Depth}{$y$}\setlength{\Height}{\Depth}%
\put(0.0000000,4.0714286){\hspace*{\Width}\raisebox{\Height}{$y$}}%
%
\settowidth{\Width}{O}\setlength{\Width}{0\Width}%
\settoheight{\Height}{O}\settodepth{\Depth}{O}\setlength{\Height}{-\Height}%
\put(0.0714286,-0.0714286){\hspace*{\Width}\raisebox{\Height}{O}}%
%
\end{picture}}%}}
\end{layer}

\begin{itemize}
\item
定数関数$f(x)=c$($c$は定数)\\
\hspace*{1zw}$(c)'=0$
\item
$f(x)=x$\\
\hspace*{1zw}$(x)'=\dlim_{z \to x}\bunsuu{z-x}{z-x}=1$
\item
$(x^2)'=2x$
\item
$(x^3)'=3x^2$
\item
一般に $(x^n)'=\hakoa{$n x^{n-1}$}$
\end{itemize}

\newslide{微分の性質(和と定数倍)}

\vspace*{18mm}

\slidepage
\vspace{2mm}

\noindent
$f(x),\ g(x)$と定数$c$について
\begin{itemize}
\item
$(f+g)'=f'+g',\ (f-g)'=f'-g'$
\item
$(c f)'=c f'$
\item
[例]\hspace{-1mm})\ $(x^2+3x+4)'=(x^2)'+(3x)'+(4)'$
$=2x+3$
\item
[問]\monban 微分せよ\seteda{80}\\
\eda{$y=2x^2-3x+2$}\\\eda{$y=\bunsuu{1}{3}x^3-\bunsuu{1}{2}x^2+2x+1$}
\end{itemize}

\mainslide{積と商の微分・記法}


\slidepage[m]
%%%%%%%%%%%%%

%%%%%%%%%%%%%%%%%%%%

\newslide{積の微分}

\vspace*{18mm}

\slidepage
\begin{itemize}
\item
\fbox{$(f\,g)'=f'\,g+f\,g'$}\hspace{1zw}{\color{red}積の微分公式}
\item
[]$\bigl(f(x)g(x)\bigr)'=\dlim_{z \to x}\bunsuu{f(z)g(z)-f(x)g(x)}{z-x}$\\
\hspace*{1zw}$=\dlim_{z \to x}\bunsuu{\bigl(f(z)-f(x)\bigr)g(z)+f(x)\bigl(g(z)-g(x)\bigr)}{z-x}$\\\hspace*{1zw}$=\dlim_{z \to x}\left(\bunsuu{f(z)-f(x)}{z-x}g(z)+f(x)\bunsuu{g(z)-g(x)}{z-x}\right)$\vspace{1mm}\\
\hspace*{1zw}$=f'(x)g(x)+f(x)g'(x)$
\end{itemize}

\newslide{積の微分の例}

\vspace*{18mm}

\slidepage
\begin{itemize}
\item
[例]$y'=\bigl((x+1)(x^2+2x+3) \bigr)'$\\
$\phn{y'}=(x+1)'(x^2+2x+3)+(x+1)(x^2+2x+3)'$\\
$\phn{y'}=(x^2+2x+3)+(x+1)(2x+2)$\\
$\phn{y'}=3x^2+6x+5$
\item
[問]\monban 積の微分公式で微分せよ.\seteda{65}\\
\eda{$y=(x+1)(x+3)$}\eda{$y=x^2(x+2)$}
\end{itemize}

\newslide{商の微分}

\vspace*{18mm}

\slidepage
\textinit

\begin{layer}{120}{0}
\addtext[-3]{8}{\ten}{\fbox{$\left(\bunsuu{f}{g}\right)'=\bunsuu{f'\,g-f\,g'}{g^2}$}\hspace{1zw}{\color{red}商の微分公式}}
\addtext[8]{8}{[例(1)]}{\large$\left(\bunsuu{2x+1}{3x+1}\right)'$}
\addtext[4]{30}{}{\large$=\bunsuu{(2x+1)'(3x+1)-(2x+1)(3x+1)'}{(3x+1)^2}$}
\addtext[3]{30}{}{\large$=\bunsuu{2(3x+1)-3(2x+1)}{(3x+1)^2}$}
\addtext[-8]{83}{}{\large$=\bunsuu{-1}{(3x+1)^2}$}
\addtext[4]{8}{[例(2)]}{\large$\left(\bunsuu{1}{x}\right)'$}
\addtext[-7]{40}{}{\large$=\bunsuu{(1)'(x)-1(x)'}{x^2}$}
\addtext[-7]{80}{}{\large$=\bunsuu{0-1}{x^2}$}
\addtext[-8]{100}{}{\large$=-\bunsuu{1}{x^2}$}
\addtext[1]{8}{問}{\monbannoadd $y=\bunsuu{x}{x+1}$を微分せよ}
\end{layer}

\addban

\newslide{$x^{p}$の微分}

\vspace*{18mm}


\begin{layer}{120}{0}
\putnotew{96}{73}{\hyperlink{para2pg10}{\fbox{\Ctab{2.5mm}{\scalebox{1}{\scriptsize $\mathstrut||\!\lhd$}}}}}
\putnotew{101}{73}{\hyperlink{para3pg1}{\fbox{\Ctab{2.5mm}{\scalebox{1}{\scriptsize $\mathstrut|\!\lhd$}}}}}
\putnotew{108}{73}{\hyperlink{para3pg9}{\fbox{\Ctab{4.5mm}{\scalebox{1}{\scriptsize $\mathstrut\!\!\lhd\!\!$}}}}}
\putnotew{115}{73}{\hyperlink{para3pg10}{\fbox{\Ctab{4.5mm}{\scalebox{1}{\scriptsize $\mathstrut\!\rhd\!$}}}}}
\putnotew{120}{73}{\hyperlink{para3pg10}{\fbox{\Ctab{2.5mm}{\scalebox{1}{\scriptsize $\mathstrut \!\rhd\!\!|$}}}}}
\putnotew{125}{73}{\hyperlink{para4pg1}{\fbox{\Ctab{2.5mm}{\scalebox{1}{\scriptsize $\mathstrut \!\rhd\!\!||$}}}}}
\putnotee{126}{73}{\scriptsize\color{blue} 10/10}
\end{layer}

\slidepage

\begin{layer}{120}{0}
\putnotee{55}{38}{\color{red}\normalsize $\sqrt{z}=w,\sqrt{x}=u$とおくと $z=w^2,x=u^2$}
\end{layer}

\begin{itemize}
\item
$n$が正の整数のとき \fbox{$(x^n)'=n x^{n-1}$}
\item
分数乗\\
\hspace*{0.5zw}$(x^{\frac{1}{2}})'=(\sqrt{x})'$
$=\dlim_{z \to x}\bunsuu{\sqrt{z}-\sqrt{x}}{z-x}$
$=\dlim_{w \to u}\bunsuu{w-u}{w^2-u^2}$\vspace{6mm}\\
\hspace*{0.5zw}$\phn{(x^{\frac{1}{2}})'}=\dlim_{w \to u}\bunsuu{1}{w+u}$
$=\bunsuu{1}{2u}$
$=\bunsuu{1}{2\sqrt{x}}$
$=\bunsuu{1}{2}x^{-\frac{1}{2}}$
\item
[問]\monban $y=x^{\frac{3}{2}}=x\sqrt{x}$を微分せよ.
\end{itemize}

\newslide{$x^{p}$の微分公式}

\vspace*{18mm}


\begin{layer}{120}{0}
\putnotew{96}{73}{\hyperlink{para3pg10}{\fbox{\Ctab{2.5mm}{\scalebox{1}{\scriptsize $\mathstrut||\!\lhd$}}}}}
\putnotew{101}{73}{\hyperlink{para4pg1}{\fbox{\Ctab{2.5mm}{\scalebox{1}{\scriptsize $\mathstrut|\!\lhd$}}}}}
\putnotew{108}{73}{\hyperlink{para4pg6}{\fbox{\Ctab{4.5mm}{\scalebox{1}{\scriptsize $\mathstrut\!\!\lhd\!\!$}}}}}
\putnotew{115}{73}{\hyperlink{para4pg7}{\fbox{\Ctab{4.5mm}{\scalebox{1}{\scriptsize $\mathstrut\!\rhd\!$}}}}}
\putnotew{120}{73}{\hyperlink{para4pg7}{\fbox{\Ctab{2.5mm}{\scalebox{1}{\scriptsize $\mathstrut \!\rhd\!\!|$}}}}}
\putnotew{125}{73}{\hyperlink{para5pg1}{\fbox{\Ctab{2.5mm}{\scalebox{1}{\scriptsize $\mathstrut \!\rhd\!\!||$}}}}}
\putnotee{126}{73}{\scriptsize\color{blue} 7/7}
\end{layer}

\slidepage
\begin{itemize}
\item
$(x^p)'=\hakoma{p x^{p-1}}$
\item
マイナス乗も同じ\\
\hspace*{1zw}$(\bunsuu{1}{x})'$
$=(x^{-1})'$
$=-x^{-2}$
$=-\bunsuu{1}{x^2}$
\item
[問]\monban 次の関数を微分せよ.\seteda{40}\\
\eda{$y=x^{\frac{1}{4}}$}\eda{$y=x^{-2}$}\eda{$y=x^{-\frac{1}{2}}$}
\end{itemize}

\mainslide{三角関数の微分}


\slidepage[m]
%%%%%%%%%%%%%

%%%%%%%%%%%%%%%%%%%%

\newslide{三角関数のグラフ}

\vspace*{18mm}

\slidepage

\begin{layer}{120}{0}
\putnotes{60}{10}{\scalebox{0.8}{%%% /Users/takatoosetsuo/Desktop/polytech22.git/202-0704/presen/fig/sincosgraph.tex 
%%% Generator=fig22202.cdy 
{\unitlength=18mm%
\begin{picture}%
(6.28,2.4)(-3.14,-1.2)%
\linethickness{0.008in}%%
\polyline(-3.14000,-0.00159)(-3.01440,-0.12685)(-2.88880,-0.25011)(-2.76320,-0.36943)%
(-2.63760,-0.48293)(-2.51200,-0.58882)(-2.38640,-0.68543)(-2.26080,-0.77124)(-2.13520,-0.84491)%
(-2.00960,-0.90526)(-1.88400,-0.95135)(-1.75840,-0.98245)(-1.63280,-0.99808)(-1.50720,-0.99798)%
(-1.38160,-0.98216)(-1.25600,-0.95086)(-1.13040,-0.90458)(-1.00480,-0.84405)(-0.87920,-0.77023)%
(-0.75360,-0.68427)(-0.62800,-0.58753)(-0.50240,-0.48153)(-0.37680,-0.36795)(-0.25120,-0.24857)%
(-0.12560,-0.12527)(0.00000,0.00000)(0.12560,0.12527)(0.25120,0.24857)(0.37680,0.36795)%
(0.50240,0.48153)(0.62800,0.58753)(0.75360,0.68427)(0.87920,0.77023)(1.00480,0.84405)%
(1.13040,0.90458)(1.25600,0.95086)(1.38160,0.98216)(1.50720,0.99798)(1.63280,0.99808)%
(1.75840,0.98245)(1.88400,0.95135)(2.00960,0.90526)(2.13520,0.84491)(2.26080,0.77124)%
(2.38640,0.68543)(2.51200,0.58882)(2.63760,0.48293)(2.76320,0.36943)(2.88880,0.25011)%
(3.01440,0.12685)(3.14000,0.00159)%
%
\polyline(-3.14000,-1.00000)(-3.01440,-0.99192)(-2.88880,-0.96822)(-2.76320,-0.92926)%
(-2.63760,-0.87566)(-2.51200,-0.80827)(-2.38640,-0.72814)(-2.26080,-0.63654)(-2.13520,-0.53491)%
(-2.00960,-0.42486)(-1.88400,-0.30811)(-1.75840,-0.18651)(-1.63280,-0.06196)(-1.50720,0.06355)%
(-1.38160,0.18807)(-1.25600,0.30962)(-1.13040,0.42630)(-1.00480,0.53626)(-0.87920,0.63777)%
(-0.75360,0.72923)(-0.62800,0.80920)(-0.50240,0.87643)(-0.37680,0.92985)(-0.25120,0.96861)%
(-0.12560,0.99212)(0.00000,1.00000)(0.12560,0.99212)(0.25120,0.96861)(0.37680,0.92985)%
(0.50240,0.87643)(0.62800,0.80920)(0.75360,0.72923)(0.87920,0.63777)(1.00480,0.53626)%
(1.13040,0.42630)(1.25600,0.30962)(1.38160,0.18807)(1.50720,0.06355)(1.63280,-0.06196)%
(1.75840,-0.18651)(1.88400,-0.30811)(2.00960,-0.42486)(2.13520,-0.53491)(2.26080,-0.63654)%
(2.38640,-0.72814)(2.51200,-0.80827)(2.63760,-0.87566)(2.76320,-0.92926)(2.88880,-0.96822)%
(3.01440,-0.99192)(3.14000,-1.00000)%
%
\polyline(-3.14000,0.00159)(-3.01440,0.12685)(-2.88880,0.25011)(-2.76320,0.36943)%
(-2.63760,0.48293)(-2.51200,0.58882)(-2.38640,0.68543)(-2.26080,0.77124)(-2.13520,0.84491)%
(-2.00960,0.90526)(-1.88400,0.95135)(-1.75840,0.98245)(-1.63280,0.99808)(-1.50720,0.99798)%
(-1.38160,0.98216)(-1.25600,0.95086)(-1.13040,0.90458)(-1.00480,0.84405)(-0.87920,0.77023)%
(-0.75360,0.68427)(-0.62800,0.58753)(-0.50240,0.48153)(-0.37680,0.36795)(-0.25120,0.24857)%
(-0.12560,0.12527)(0.00000,0.00000)(0.12560,-0.12527)(0.25120,-0.24857)(0.37680,-0.36795)%
(0.50240,-0.48153)(0.62800,-0.58753)(0.75360,-0.68427)(0.87920,-0.77023)(1.00480,-0.84405)%
(1.13040,-0.90458)(1.25600,-0.95086)(1.38160,-0.98216)(1.50720,-0.99798)(1.63280,-0.99808)%
(1.75840,-0.98245)(1.88400,-0.95135)(2.00960,-0.90526)(2.13520,-0.84491)(2.26080,-0.77124)%
(2.38640,-0.68543)(2.51200,-0.58882)(2.63760,-0.48293)(2.76320,-0.36943)(2.88880,-0.25011)%
(3.01440,-0.12685)(3.14000,-0.00159)%
%
\polyline(-3.14000,1.00000)(-3.01440,0.99192)(-2.88880,0.96822)(-2.76320,0.92926)%
(-2.63760,0.87566)(-2.51200,0.80827)(-2.38640,0.72814)(-2.26080,0.63654)(-2.13520,0.53491)%
(-2.00960,0.42486)(-1.88400,0.30811)(-1.75840,0.18651)(-1.63280,0.06196)(-1.50720,-0.06355)%
(-1.38160,-0.18807)(-1.25600,-0.30962)(-1.13040,-0.42630)(-1.00480,-0.53626)(-0.87920,-0.63777)%
(-0.75360,-0.72923)(-0.62800,-0.80920)(-0.50240,-0.87643)(-0.37680,-0.92985)(-0.25120,-0.96861)%
(-0.12560,-0.99212)(0.00000,-1.00000)(0.12560,-0.99212)(0.25120,-0.96861)(0.37680,-0.92985)%
(0.50240,-0.87643)(0.62800,-0.80920)(0.75360,-0.72923)(0.87920,-0.63777)(1.00480,-0.53626)%
(1.13040,-0.42630)(1.25600,-0.30962)(1.38160,-0.18807)(1.50720,-0.06355)(1.63280,0.06196)%
(1.75840,0.18651)(1.88400,0.30811)(2.00960,0.42486)(2.13520,0.53491)(2.26080,0.63654)%
(2.38640,0.72814)(2.51200,0.80827)(2.63760,0.87566)(2.76320,0.92926)(2.88880,0.96822)%
(3.01440,0.99192)(3.14000,1.00000)%
%
\polyline(0.02778,-1.00000)(-0.02778,-1.00000)%
%
\settowidth{\Width}{$-1$}\setlength{\Width}{-1\Width}%
\settoheight{\Height}{$-1$}\settodepth{\Depth}{$-1$}\setlength{\Height}{-0.5\Height}\setlength{\Depth}{0.5\Depth}\addtolength{\Height}{\Depth}%
\put(-0.0555556,-1.0000000){\hspace*{\Width}\raisebox{\Height}{$-1$}}%
%
\polyline(0.02778,1.00000)(-0.02778,1.00000)%
%
\settowidth{\Width}{$1$}\setlength{\Width}{-1\Width}%
\settoheight{\Height}{$1$}\settodepth{\Depth}{$1$}\setlength{\Height}{-0.5\Height}\setlength{\Depth}{0.5\Depth}\addtolength{\Height}{\Depth}%
\put(-0.0555556,1.0000000){\hspace*{\Width}\raisebox{\Height}{$1$}}%
%
\polyline(-1.57080,0.02778)(-1.57080,-0.02778)%
%
\settowidth{\Width}{$-\frac{\pi}{2}$}\setlength{\Width}{-0.5\Width}%
\settoheight{\Height}{$-\frac{\pi}{2}$}\settodepth{\Depth}{$-\frac{\pi}{2}$}\setlength{\Height}{-\Height}%
\put(-1.5700000,-0.0555556){\hspace*{\Width}\raisebox{\Height}{$-\frac{\pi}{2}$}}%
%
\polyline(1.57080,0.02778)(1.57080,-0.02778)%
%
\settowidth{\Width}{$\frac{\pi}{2}$}\setlength{\Width}{-0.5\Width}%
\settoheight{\Height}{$\frac{\pi}{2}$}\settodepth{\Depth}{$\frac{\pi}{2}$}\setlength{\Height}{-\Height}%
\put(1.5700000,-0.0555556){\hspace*{\Width}\raisebox{\Height}{$\frac{\pi}{2}$}}%
%
\settowidth{\Width}{[1]}\setlength{\Width}{-0.5\Width}%
\settoheight{\Height}{[1]}\settodepth{\Depth}{[1]}\setlength{\Height}{\Depth}%
\put(-2.9600000,1.1577778){\hspace*{\Width}\raisebox{\Height}{[1]}}%
%
\settowidth{\Width}{[2]}\setlength{\Width}{-0.5\Width}%
\settoheight{\Height}{[2]}\settodepth{\Depth}{[2]}\setlength{\Height}{\Depth}%
\put(-1.5200000,1.1577778){\hspace*{\Width}\raisebox{\Height}{[2]}}%
%
\settowidth{\Width}{[3]}\setlength{\Width}{-0.5\Width}%
\settoheight{\Height}{[3]}\settodepth{\Depth}{[3]}\setlength{\Height}{\Depth}%
\put(0.2800000,1.1577778){\hspace*{\Width}\raisebox{\Height}{[3]}}%
%
\settowidth{\Width}{[4]}\setlength{\Width}{-0.5\Width}%
\settoheight{\Height}{[4]}\settodepth{\Depth}{[4]}\setlength{\Height}{\Depth}%
\put(1.5400000,1.1677778){\hspace*{\Width}\raisebox{\Height}{[4]}}%
%
\polyline(-3.14000,0.00000)(3.14000,0.00000)%
%
\polyline(0.00000,-1.20000)(0.00000,1.20000)%
%
\settowidth{\Width}{$x$}\setlength{\Width}{0\Width}%
\settoheight{\Height}{$x$}\settodepth{\Depth}{$x$}\setlength{\Height}{-0.5\Height}\setlength{\Depth}{0.5\Depth}\addtolength{\Height}{\Depth}%
\put(3.1677778,0.0000000){\hspace*{\Width}\raisebox{\Height}{$x$}}%
%
\settowidth{\Width}{$y$}\setlength{\Width}{-0.5\Width}%
\settoheight{\Height}{$y$}\settodepth{\Depth}{$y$}\setlength{\Height}{\Depth}%
\put(0.0000000,1.2277778){\hspace*{\Width}\raisebox{\Height}{$y$}}%
%
\settowidth{\Width}{O}\setlength{\Width}{-1\Width}%
\settoheight{\Height}{O}\settodepth{\Depth}{O}\setlength{\Height}{-\Height}%
\put(-0.0277778,-0.0277778){\hspace*{\Width}\raisebox{\Height}{O}}%
%
\end{picture}}%}}
\end{layer}

\vspace{40mm}
\begin{itemize}
\item
[問]\monban 図は\\
\hspace*{2zw}$y=\sin x,y=\cos x, y=-\sin x, y=-\cos x$\\
のグラフである.アプリを用いて,関数の番号を答えよ.
\end{itemize}
%%%%%%%%%%%%%

%%%%%%%%%%%%%%%%%%%%


\newslide{$\sin x,\cos x$の微分}

\vspace*{18mm}


\begin{layer}{120}{0}
\putnotew{96}{73}{\hyperlink{para4pg1}{\fbox{\Ctab{2.5mm}{\scalebox{1}{\scriptsize $\mathstrut||\!\lhd$}}}}}
\putnotew{101}{73}{\hyperlink{para5pg1}{\fbox{\Ctab{2.5mm}{\scalebox{1}{\scriptsize $\mathstrut|\!\lhd$}}}}}
\putnotew{108}{73}{\hyperlink{para5pg2}{\fbox{\Ctab{4.5mm}{\scalebox{1}{\scriptsize $\mathstrut\!\!\lhd\!\!$}}}}}
\putnotew{115}{73}{\hyperlink{para5pg3}{\fbox{\Ctab{4.5mm}{\scalebox{1}{\scriptsize $\mathstrut\!\rhd\!$}}}}}
\putnotew{120}{73}{\hyperlink{para5pg3}{\fbox{\Ctab{2.5mm}{\scalebox{1}{\scriptsize $\mathstrut \!\rhd\!\!|$}}}}}
\putnotew{125}{73}{\hyperlink{para6pg1}{\fbox{\Ctab{2.5mm}{\scalebox{1}{\scriptsize $\mathstrut \!\rhd\!\!||$}}}}}
\putnotee{126}{73}{\scriptsize\color{blue} 3/3}
\end{layer}

\slidepage
\begin{itemize}
\item
[問]\monbannoadd アプリを用いて導関数を求めよ.\seteda{50}\\
\eda{$y=\sin x$}\eda{$y=\cos x$}
\item
微分公式\\
\hspace*{2zw}\fbox{$(\sin x)'=\cos x,\ (\cos x)'=-\sin x$}
\addban
\item
[問]\monban 次の関数を微分せよ\\
\hspace*{10mm}$y=2\sin x-3\cos x$
\end{itemize}

\newslide{$\tan x$の微分}

\vspace*{18mm}


\begin{layer}{120}{0}
\putnotew{96}{73}{\hyperlink{para5pg3}{\fbox{\Ctab{2.5mm}{\scalebox{1}{\scriptsize $\mathstrut||\!\lhd$}}}}}
\putnotew{101}{73}{\hyperlink{para6pg1}{\fbox{\Ctab{2.5mm}{\scalebox{1}{\scriptsize $\mathstrut|\!\lhd$}}}}}
\putnotew{108}{73}{\hyperlink{para6pg4}{\fbox{\Ctab{4.5mm}{\scalebox{1}{\scriptsize $\mathstrut\!\!\lhd\!\!$}}}}}
\putnotew{115}{73}{\hyperlink{para6pg5}{\fbox{\Ctab{4.5mm}{\scalebox{1}{\scriptsize $\mathstrut\!\rhd\!$}}}}}
\putnotew{120}{73}{\hyperlink{para6pg5}{\fbox{\Ctab{2.5mm}{\scalebox{1}{\scriptsize $\mathstrut \!\rhd\!\!|$}}}}}
\putnotew{125}{73}{\hyperlink{para7pg1}{\fbox{\Ctab{2.5mm}{\scalebox{1}{\scriptsize $\mathstrut \!\rhd\!\!||$}}}}}
\putnotee{126}{73}{\scriptsize\color{blue} 5/5}
\end{layer}

\slidepage
{\color{red}\large

\begin{layer}{120}{0}
\putnotee{60}{10}{$\tan x=\bunsuu{\sin x}{\cos x}$}
\putnotee{60}{20}{$\cos^2 x=(\cos x)^2$}
\end{layer}

}
\begin{itemize}
\item
\fbox{$(\tan x)'=\bunsuu{1}{\cos^2 x}$}
\item
[]$(\tan x)'=\bigl(\bunsuu{\sin x}{\cos x}\bigr)'$\\
$\phn{(\tan x)'}=\bunsuu{(\sin x)'(\cos x)-(\sin x)(\cos x)'}{\cos^2 x}$\\
$\phn{(\tan x)'}=\bunsuu{(\cos x \cos x)-\sin x(-\sin x)'}{\cos^2 x}$\\
$\phn{(\tan x)'}=\bunsuu{\cos^2 x+\sin^2 x}{\cos^2 x}=\bunsuu{1}{\cos^2 x}$\\
\end{itemize}

\newslide{質問}

\vspace*{18mm}

\slidepage
\begin{itemize}
\item
[問]\monban 次の関数を微分せよ\seteda{100}\\
\eda{$y=\sin x \cos x$}\\
\eda{$y=\sin^2 x(=\sin x \sin x)$}\\
\eda{$y=x \tan x$}\\
\eda{$y=\tan x-x$}
\end{itemize}
\label{pageend}\mbox{}

\end{document}

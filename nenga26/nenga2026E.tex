\documentclass[10pt]{article}

\usepackage{geometry}
\geometry{paperwidth=148mm,paperheight=100mm}

%\usepackage{luatexja}

\usepackage{hyperref} 
\usepackage{graphicx,color}
%\usepackage{amsmath,amssymb}
\usepackage{qrcode}
\usepackage{pict2e}
\usepackage{ketpic2e}
\usepackage{ketlayer2e}

\setmargin{10}{10}{10}{10}

\pagestyle{empty}

\begin{document}

\begin{layer}{80}{0}
\putnotes{46}{-7}{\includegraphics[width=100mm]{fig/2026nengaE2.png}}
\putnotese{96}{-6}{\includegraphics[width=24mm]{fig/fukidasiE.png}}
\putnotese{96}{12}{\includegraphics[width=35mm]{fig/JTIIstp3.png}}
%\putnotee{93}{43}{setsuotakato@gmail.com}
\putnotee{105}{47}{Setsuo Takato}
\end{layer}


\vspace{49mm}

\noindent
%\begin{minipage}[t]{140mm}
It’s been a full 12-year cycle since I started KeTCindy in 2014. I feel like I've finally hit my stride!
Lately, I’ve been working on solving Wasan geometry with Maxima. The top-right image shows a web app I built to visualize this. It lets you explore the link between a triangle's inradius and the three circles inside. Give it a try!% at the link below!

%Happy Year of the Horse!
%
%It’s been a full 12-year cycle since I started KeTCindy in 2014. I feel like I've finally hit my stride!
%
%Lately, I’ve been working on solving Wasan geometry with Maxima. The top-right image shows a web app I built to visualize this. It lets you explore the link between a triangle's inradius and the three circles inside. Give it a try at the link below!\ \ 
\noindent
\hspace{3mm}{\small \url{https://s-takato.github.io/specialclass/wasan/offline/jpn2outdgjsoffL.html}}\\
%\vspace{2mm}
Wishing you good health and happiness in the new year. 
\end{document}


\documentclass[10pt]{article}

\usepackage{geometry}
\geometry{paperwidth=148mm,paperheight=100mm}

\usepackage{luatexja}

\usepackage{hyperref} 
\usepackage{graphicx,color}
%\usepackage{amsmath,amssymb}
\usepackage{qrcode}
\usepackage{pict2e}
\usepackage{ketpic2e}
\usepackage{ketlayer2e}
%\usepackage{lmodern}

%\usepackage{otf}

%\newcommand*{\myfont}[4]{{%
%  #1\fontseries{#2}\fontshape{t#3}\selectfont
%  #4
%}}
%\definecolor{slidecolora}{cmyk}{0.98,0.13,0,0.43}

\setmargin{10}{10}{10}{10}

\pagestyle{empty}

\begin{document}


\begin{layer}{80}{0}
\putnotes{46}{-7}{\includegraphics[width=100mm]{fig/2026nengaJv2.png}}
\putnotese{96}{-8}{\includegraphics[width=20mm]{fig/fukidasiJ.png}}
\putnotese{96}{10}{\includegraphics[width=35mm]{fig/JTIIstp3.png}}
\putnotee{97}{43}{{\small setsuotakato@gmail.com}}
\putnotee{105}{47}{髙遠 節夫・晴子}
\end{layer}

%\begin{layer}{100}{0}
%\putnotee{77}{78}{\fbox{\scriptsize メールをいただければ来年はPDFでお送りします}}
%\dashboxframese{75}{73}{58}{7}{\raisebox{6mm}{\scriptsize メールをいただければ来年はPDFでお送りします}}
%\end{layer}


\vspace{49mm}

\noindent
%\begin{minipage}[t]{140mm}
今年は午年.2014年にKeTCindyの開発を始めてから12年経って,\\ようやく少し跳ぶことができたかと
思っています.最近は和算の幾何をMaximaで解くことをしていて,右上の図はそのために作ったHTMLです.三角形の内接円の半径と中の3円の半径の和の関係を考えてもらうもので,以下に行けば体験できます.\\
\ \ {\small \url{https://s-takato.github.io/specialclass/wasan/offline/jpn2outdgjsoffL.html}}\vspace{2mm}

%\noindent
春の陽光のような 暖かな一年になりますように.%\vspace{2mm}


%\end{minipage}


\end{document}

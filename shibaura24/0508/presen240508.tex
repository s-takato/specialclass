%%% title presen240508
\documentclass[landscape,10pt]{ujarticle}
\usepackage{pict2e}
\usepackage{ketpic2e,ketlayer2e}
\special{papersize=\the\paperwidth,\the\paperheight}
\usepackage{ketslide}
\usepackage{amsmath,amssymb}
\usepackage{bm,enumerate}
\usepackage[dvipdfmx]{graphicx}
\usepackage{color}
\usepackage[dvipdfmx,colorlinks=true,linkcolor=blue,filecolor=blue]{hyperref}
\usepackage{emath,emathE,emathMw}

\newcommand{\monthday}{0509}

\definecolor{slidecolora}{cmyk}{0.98,0.13,0,0.43}
\definecolor{slidecolorb}{cmyk}{0.2,0,0,0}
\definecolor{slidecolorc}{cmyk}{0.2,0,0,0}
\definecolor{slidecolord}{cmyk}{0.2,0,0,0}
\definecolor{slidecolore}{cmyk}{0,0,0,0.5}
\definecolor{slidecolorf}{cmyk}{0,0,0,0.5}
\definecolor{slidecolori}{cmyk}{0.98,0.13,0,0.43}
\def\setthin#1{\def\thin{#1}}
\setthin{0}
\newcounter{pagectr}
\setcounter{pagectr}{1}
\newcommand{\slidepage}[1][\monthday-]{%
\setcounter{ketpicctra}{18}%

\begin{layer}{118}{0}
\putnotew{130}{-\theketpicctra.05}{\small#1\thepage/\pageref{pageend}}
\end{layer}

}

\setmargin{25}{145}{15}{100}

\ketslideinit

\pagestyle{empty}

\begin{document}

\begin{layer}{120}{0}
\putnotese{0}{0}{{\Large\bf
\color[cmyk]{1,1,0,0}

\begin{layer}{120}{0}
{\Huge \putnotes{60}{20}{インストール状況}}
\putnotes{60}{50}{高遠節夫}
\end{layer}

}
}
\end{layer}

\def\mainslidetitley{22}
\def\ketcletter{slidecolora}
\def\ketcbox{slidecolorb}
\def\ketdbox{slidecolorc}
\def\ketcframe{slidecolord}
\def\ketcshadow{slidecolore}
\def\ketdshadow{slidecolorf}
\def\slidetitlex{6}
\def\slidetitlesize{1.3}
\def\mketcletter{slidecolori}
\def\mketcbox{yellow}
\def\mketdbox{yellow}
\def\mketcframe{yellow}
\def\mslidetitlex{62}
\def\mslidetitlesize{2}

\color{black}
\Large\bf\boldmath
\addtocounter{page}{-1}

\renewcommand{\slidepage}[1][s]{%
\setcounter{ketpicctra}{18}%
\if#1m \setcounter{ketpicctra}{1}\fi
\hypersetup{linkcolor=black}%
\begin{layer}{118}{0}
\putnotee{115}{-\theketpicctra.05}{\small\monthday-\thepage/\pageref{pageend}}
\end{layer}\hypersetup{linkcolor=blue}
}
%%%%%%%%%%%%

%%%%%%%%%%%%%%%%%%%%

\newslide{GCとKeTLMSによるやりとり}

\vspace*{18mm}

\slidepage
\textinit

\begin{layer}{120}{0}
\addtext[0]{8}{\ban{1}}{GCの質問のリンクをクリック}
\end{layer}

%%%%%%%%%%%%

%%%%%%%%%%%%%%%%%%%%


\sameslide

\vspace*{18mm}

\slidepage
\textinit

\begin{layer}{120}{0}
\addtext[0]{8}{\ban{1}}{GCの質問のリンクをクリック}
\addtext{8}{\ban{2}}{課題を埋め込んだKeTLMSが立ち上がる}
\end{layer}


\sameslide

\vspace*{18mm}

\slidepage
\textinit

\begin{layer}{120}{0}
\addtext[0]{8}{\ban{1}}{GCの質問のリンクをクリック}
\addtext{8}{\ban{2}}{課題を埋め込んだKeTLMSが立ち上がる}
\addtext{8}{\ban{3}}{自分の番号を入れて確認,OKを押すと入力欄1に問題が入る}
\end{layer}


\sameslide

\vspace*{18mm}

\slidepage
\textinit

\begin{layer}{120}{0}
\addtext[0]{8}{\ban{1}}{GCの質問のリンクをクリック}
\addtext{8}{\ban{2}}{課題を埋め込んだKeTLMSが立ち上がる}
\addtext{8}{\ban{3}}{自分の番号を入れて確認,OKを押すと入力欄1に問題が入る}
\addtext[8]{8}{\ban{4}}{入力欄2に解答を入れてページを進める}
\end{layer}


\sameslide

\vspace*{18mm}

\slidepage
\textinit

\begin{layer}{120}{0}
\addtext[0]{8}{\ban{1}}{GCの質問のリンクをクリック}
\addtext{8}{\ban{2}}{課題を埋め込んだKeTLMSが立ち上がる}
\addtext{8}{\ban{3}}{自分の番号を入れて確認,OKを押すと入力欄1に問題が入る}
\addtext[8]{8}{\ban{4}}{入力欄2に解答を入れてページを進める}
\addtext{8}{\ban{5}}{Recを押すと全ての解答が入力欄3に入る}
\end{layer}


\sameslide

\vspace*{18mm}

\slidepage
\textinit

\begin{layer}{120}{0}
\addtext[0]{8}{\ban{1}}{GCの質問のリンクをクリック}
\addtext{8}{\ban{2}}{課題を埋め込んだKeTLMSが立ち上がる}
\addtext{8}{\ban{3}}{自分の番号を入れて確認,OKを押すと入力欄1に問題が入る}
\addtext[8]{8}{\ban{4}}{入力欄2に解答を入れてページを進める}
\addtext{8}{\ban{5}}{Recを押すと全ての解答が入力欄3に入る}
\addtext{8}{\ban{6}}{入力欄3で,すべてを選択してコピーする}
\end{layer}


\sameslide

\vspace*{18mm}

\slidepage
\textinit

\begin{layer}{120}{0}
\addtext[0]{8}{\ban{1}}{GCの質問のリンクをクリック}
\addtext{8}{\ban{2}}{課題を埋め込んだKeTLMSが立ち上がる}
\addtext{8}{\ban{3}}{自分の番号を入れて確認,OKを押すと入力欄1に問題が入る}
\addtext[8]{8}{\ban{4}}{入力欄2に解答を入れてページを進める}
\addtext{8}{\ban{5}}{Recを押すと全ての解答が入力欄3に入る}
\addtext{8}{\ban{6}}{入力欄3で,すべてを選択してコピーする}
\addtext{8}{\ban{7}}{GCの回答欄にペーストして送信を押す}
\end{layer}


\newslide{授業開始アンケート}

\vspace*{18mm}

\slidepage
\textinit

\begin{layer}{120}{0}
\addtext{8}{課題\monbannoadd}{答えてください}
\addban
\end{layer}

%%%%%%%%%%%%

%%%%%%%%%%%%%%%%%%%%


\newslide{KeTLMSの利用}

\vspace*{18mm}

\slidepage
\begin{itemize}
\item
普通の数式(2次元記法)は見やすい.\\
 $\dfrac{4}{9}$,\ $\sqrt{7}$,\ $5^3$
\end{itemize}
%%%%%%%%%%%%

%%%%%%%%%%%%%%%%%%%%


\sameslide

\vspace*{18mm}

\slidepage
\begin{itemize}
\item
普通の数式(2次元記法)は見やすい.\\
 $\dfrac{4}{9}$,\ $\sqrt{7}$,\ $5^3$
\item
しかし,オンラインでのやりとりには向かない\\
\end{itemize}

\sameslide

\vspace*{18mm}

\slidepage
\begin{itemize}
\item
普通の数式(2次元記法)は見やすい.\\
 $\dfrac{4}{9}$,\ $\sqrt{7}$,\ $5^3$
\item
しかし,オンラインでのやりとりには向かない\\
 =>1次元記法がいいが数式の意味がわかりいくい
\end{itemize}

\sameslide

\vspace*{18mm}

\slidepage
\begin{itemize}
\item
普通の数式(2次元記法)は見やすい.\\
 $\dfrac{4}{9}$,\ $\sqrt{7}$,\ $5^3$
\item
しかし,オンラインでのやりとりには向かない\\
 =>1次元記法がいいが数式の意味がわかりいくい
\item
そこで数式表示アプリをKeTMathを作った\\
\item
さらに,課題をやりとりするKeTLMSを作った
\end{itemize}

\sameslide

\vspace*{18mm}

\slidepage
\begin{itemize}
\item
普通の数式(2次元記法)は見やすい.\\
 $\dfrac{4}{9}$,\ $\sqrt{7}$,\ $5^3$
\item
しかし,オンラインでのやりとりには向かない\\
 =>1次元記法がいいが数式の意味がわかりいくい
\item
そこで数式表示アプリをKeTMathを作った\\
 ・1次元数式を入力すると即時に2次元数式を表示
\item
さらに,課題をやりとりするKeTLMSを作った
\end{itemize}

\newslide{KeTMathルール}

\vspace*{18mm}

%%repeat=4
\slidepage
\begin{itemize}
\item
\Ltab{8zw}{分数 (fraction)}$\dfrac{a}{b}\ \Longleftrightarrow$\ \verb|fr(a,b)|
\item
\Ltab{8zw}{割り算}$a\div b\ \Longleftrightarrow$\ \verb|a {\div} b|\\
\hfill(あまり使わない)
\item
\Ltab{6zw}{掛け算}$ab\ \Longleftrightarrow$\ \verb|ab|
\item
\Ltab{10zw}{べき乗}$a^b\ \Longleftrightarrow$\ \verb|a^(b)|
\item
\Ltab{12zw}{平方根 (square root)}$\sqrt{a}\ \Longleftrightarrow$\ \verb|sq(a)|
\item
\Ltab{10zw}{円周率}$\pi\ \Longleftrightarrow$\ \verb|pi|
\end{itemize}
%%%%%%%%%%%%

%%%%%%%%%%%%%%%%%%%%

\newslide{KeTLMSの使い方}

\vspace*{18mm}

\slidepage
\textinit

\begin{layer}{120}{0}
\addtext{8}{例}{2次元の数式を表示しよう}
\addtext{12}{\seteda{50}}{
\eda{sq(2)}\eda{(a+b)/(c+d)}}
\end{layer}

%%%%%%%%%%%%

%%%%%%%%%%%%%%%%%%%%


\sameslide

\vspace*{18mm}

\slidepage
\textinit

\begin{layer}{120}{0}
\addtext{8}{例}{2次元の数式を表示しよう}
\addtext{12}{\seteda{50}}{
\eda{sq(2)}\eda{(a+b)/(c+d)}}
\addtext{8}{\ten}{数式表示}
\addtext{12}{\seteda{50}}{
\eda{$\sqrt{2}$}\eda{$\dfrac{a+b}{c+d}$}}
\end{layer}


\newslide{KeTMathの練習}

\vspace*{18mm}

\slidepage
\textinit

\begin{layer}{120}{0}
\addtext{8}{課題\monbannoadd}{2次元の数式で表示しよう}
\addtext{10}{\seteda{50}}{
\eda{fr(1+4,3)}\eda{a+b/c+d}\\
\eda{sq(2)/2}\eda{pir\^\,\hspace{-1mm}(2)}}
\end{layer}

%%%%%%%%%%%%

%%%%%%%%%%%%%%%%%%%%


\sameslide

\vspace*{18mm}

\slidepage
\textinit

\begin{layer}{120}{0}
\addtext{8}{課題\monbannoadd}{2次元の数式で表示しよう}
\addtext{10}{\seteda{50}}{
\eda{fr(1+4,3)}\eda{a+b/c+d}\\
\eda{sq(2)/2}\eda{pir\^\,\hspace{-1mm}(2)}}
\addban
\addtext[8]{8}{解答}{}
\addtext{10}{\seteda{50}}{
\eda{$\dfrac{1+4}{3}$}\eda{$a+\dfrac{b}{c}+d$}\\
\eda{$\dfrac{\sqrt{2}}{2}$}\eda{$\pi r^2$}}
\end{layer}


\newslide{KeTLMSの練習2}

\vspace*{18mm}

\slidepage
\textinit

\begin{layer}{120}{0}
\addtext{8}{課題\monbannoadd}{次の式をKeTMath数式で書け}
\addtext{12}{\seteda{50}}{
\eda{$-\dfrac{3}{5}$}\eda{$\dfrac{xy}{x+y}$}\\
\eda{$\sqrt{3}-\sqrt{2}$}\eda{$\dfrac{\pi}{2}$}}
\addban
\addtext[16]{8}{課題\monbannoadd}{次の式をKeTMath数式で書け}
\addtext{12}{\seteda{50}}{
\eda{$2x^2+3x+1$}\eda{$3x-4=0$}\\
\eda{$\dfrac{1}{\sqrt{3}}=\dfrac{\sqrt{3}}{3}$}\eda{$\sin x+\cos x$}}
\end{layer}

\addban
\label{pageend}\mbox{}

\end{document}

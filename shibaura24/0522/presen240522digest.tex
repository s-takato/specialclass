%%% title presen240522
\documentclass[landscape,10pt]{ujarticle}
\usepackage{pict2e}
\usepackage{ketpic2e,ketlayer2e}
\special{papersize=\the\paperwidth,\the\paperheight}
\usepackage{ketslide}
\usepackage{amsmath,amssymb}
\usepackage{bm,enumerate}
\usepackage[dvipdfmx]{graphicx}
\usepackage{color}
\usepackage[dvipdfmx,colorlinks=true,linkcolor=blue,filecolor=blue]{hyperref}
\usepackage{emath,emathE,emathMw}

\newcommand{\monthday}{0522}

\definecolor{slidecolora}{cmyk}{0.98,0.13,0,0.43}
\definecolor{slidecolorb}{cmyk}{0.2,0,0,0}
\definecolor{slidecolorc}{cmyk}{0.2,0,0,0}
\definecolor{slidecolord}{cmyk}{0.2,0,0,0}
\definecolor{slidecolore}{cmyk}{0,0,0,0.5}
\definecolor{slidecolorf}{cmyk}{0,0,0,0.5}
\definecolor{slidecolori}{cmyk}{0.98,0.13,0,0.43}
\def\setthin#1{\def\thin{#1}}
\setthin{0}
\newcounter{pagectr}
\setcounter{pagectr}{1}
\newcommand{\slidepage}[1][\monthday-]{%
\setcounter{ketpicctra}{18}%

\begin{layer}{118}{0}
\putnotew{130}{-\theketpicctra.05}{\small#1\thepage/\pageref{pageend}}
\end{layer}

}

\setmargin{25}{145}{15}{100}

\ketslideinit

\pagestyle{empty}

\begin{document}

\begin{layer}{120}{0}
\putnotese{0}{0}{{\Large\bf
\color[cmyk]{1,1,0,0}

\begin{layer}{120}{0}
{\Huge \putnotes{60}{20}{インストール状況}}
\putnotes{60}{50}{高遠節夫}
\end{layer}

}
}
\end{layer}

\def\mainslidetitley{22}
\def\ketcletter{slidecolora}
\def\ketcbox{slidecolorb}
\def\ketdbox{slidecolorc}
\def\ketcframe{slidecolord}
\def\ketcshadow{slidecolore}
\def\ketdshadow{slidecolorf}
\def\slidetitlex{6}
\def\slidetitlesize{1.3}
\def\mketcletter{slidecolori}
\def\mketcbox{yellow}
\def\mketdbox{yellow}
\def\mketcframe{yellow}
\def\mslidetitlex{62}
\def\mslidetitlesize{2}

\color{black}
\Large\bf\boldmath
\addtocounter{page}{-1}

\renewcommand{\slidepage}[1][s]{%
\setcounter{ketpicctra}{18}%
\if#1m \setcounter{ketpicctra}{1}\fi
\hypersetup{linkcolor=black}%
\begin{layer}{118}{0}
\putnotee{115}{-\theketpicctra.05}{\small\monthday-\thepage/\pageref{pageend}}
\end{layer}\hypersetup{linkcolor=blue}
}
%%%%%%%%%%%%

%%%%%%%%%%%%%%%%%%%%

\mainslide{各種ソフトのインストール}


%%%%%%%%%%%%

%%%%%%%%%%%%%%%%%%%%

\newslide{KeTLMSの教材例}

\vspace*{18mm}

\slidepage
\textinit

\begin{layer}{120}{0}
\putnotese{80}{15}{\scalebox{0.8}{%%% /Users/takatoosetsuo/specialclass.git/shibaura24/0522/fig/radian.tex 
%%% Generator=presen240522.cdy 
{\unitlength=2cm%
\begin{picture}%
(2.4,2.4)(-1.2,-1.2)%
\linethickness{0.008in}%%
\Large\bf\boldmath%
\small%
\linethickness{0.016in}%%
\polyline(1,0)(0.992,0.125)(0.969,0.249)(0.93,0.368)(0.876,0.482)(0.809,0.588)(0.729,0.685)%
(0.637,0.771)(0.536,0.844)(0.426,0.905)(0.309,0.951)(0.187,0.982)(0.063,0.998)(-0.063,0.998)%
(-0.187,0.982)(-0.309,0.951)(-0.426,0.905)(-0.536,0.844)(-0.637,0.771)(-0.729,0.685)%
(-0.809,0.588)(-0.876,0.482)(-0.93,0.368)(-0.969,0.249)(-0.992,0.125)(-1,0)(-0.992,-0.125)%
(-0.969,-0.249)(-0.93,-0.368)(-0.876,-0.482)(-0.809,-0.588)(-0.729,-0.685)(-0.637,-0.771)%
(-0.536,-0.844)(-0.426,-0.905)(-0.309,-0.951)(-0.187,-0.982)(-0.063,-0.998)(0.063,-0.998)%
(0.187,-0.982)(0.309,-0.951)(0.426,-0.905)(0.536,-0.844)(0.637,-0.771)(0.729,-0.685)%
(0.809,-0.588)(0.876,-0.482)(0.93,-0.368)(0.969,-0.249)(0.992,-0.125)(1,0)%
%
\linethickness{0.008in}%%
{%
\color[cmyk]{0,1,1,0}%
\linethickness{0.016in}%%
\polyline(1,0)(1,0.01)(1,0.02)(1,0.03)(0.999,0.04)(0.999,0.05)(0.998,0.06)(0.998,0.07)%
(0.997,0.08)(0.996,0.09)(0.995,0.1)(0.994,0.11)(0.993,0.12)(0.991,0.13)(0.99,0.14)%
(0.989,0.15)(0.987,0.16)(0.985,0.17)(0.984,0.18)(0.982,0.19)(0.98,0.2)(0.978,0.21)%
(0.976,0.219)(0.973,0.229)(0.971,0.239)(0.969,0.249)(0.966,0.258)(0.963,0.268)(0.961,0.278)%
(0.958,0.287)(0.955,0.297)(0.952,0.307)(0.949,0.316)(0.945,0.326)(0.942,0.335)(0.939,0.345)%
(0.935,0.354)(0.932,0.363)(0.928,0.373)(0.924,0.382)(0.92,0.391)(0.916,0.401)(0.912,0.41)%
(0.908,0.419)(0.904,0.428)(0.899,0.437)(0.895,0.446)(0.89,0.455)(0.886,0.464)(0.881,0.473)%
(0.876,0.482)%
%
\linethickness{0.008in}%%
}%
\linethickness{0.016in}%%
\polyline(0,0)(0.88,0.482)%
%
\linethickness{0.008in}%%
\polyline(0.5,0)(0.496,0.063)(0.484,0.124)(0.465,0.184)(0.439,0.24)%
%
\settowidth{\Width}{$\theta$}\setlength{\Width}{-0.5\Width}%
\settoheight{\Height}{$\theta$}\settodepth{\Depth}{$\theta$}\setlength{\Height}{-0.5\Height}\setlength{\Depth}{0.5\Depth}\addtolength{\Height}{\Depth}%
\put(  0.580,  0.150){\hspace*{\Width}\raisebox{\Height}{$\theta$}}%
%
\settowidth{\Width}{$l=\theta$}\setlength{\Width}{0\Width}%
\settoheight{\Height}{$l=\theta$}\settodepth{\Depth}{$l=\theta$}\setlength{\Height}{\Depth}%
\put(  0.995,  0.275){\hspace*{\Width}\raisebox{\Height}{$l=\theta$}}%
%
{%
\color[cmyk]{0,1,1,0}%
\settowidth{\Width}{$\theta=\dfrac{l}{r}$}\setlength{\Width}{0\Width}%
\settoheight{\Height}{$\theta=\dfrac{l}{r}$}\settodepth{\Depth}{$\theta=\dfrac{l}{r}$}\setlength{\Height}{-0.5\Height}\setlength{\Depth}{0.5\Depth}\addtolength{\Height}{\Depth}%
\put(  0.565,  1.110){\hspace*{\Width}\raisebox{\Height}{$\theta=\dfrac{l}{r}$}}%
%
}%
\polyline(-1,-0.025)(-1,0.025)%
%
\settowidth{\Width}{$-1$}\setlength{\Width}{-1\Width}%
\settoheight{\Height}{$-1$}\settodepth{\Depth}{$-1$}\setlength{\Height}{-\Height}%
\put( -1.025, -0.050){\hspace*{\Width}\raisebox{\Height}{$-1$}}%
%
\polyline(1,-0.025)(1,0.025)%
%
\settowidth{\Width}{$1$}\setlength{\Width}{0\Width}%
\settoheight{\Height}{$1$}\settodepth{\Depth}{$1$}\setlength{\Height}{-\Height}%
\put(  1.025, -0.050){\hspace*{\Width}\raisebox{\Height}{$1$}}%
%
\polyline(-0.025,-1)(0.025,-1)%
%
\settowidth{\Width}{$-1$}\setlength{\Width}{-1\Width}%
\settoheight{\Height}{$-1$}\settodepth{\Depth}{$-1$}\setlength{\Height}{-\Height}%
\put( -0.050, -1.025){\hspace*{\Width}\raisebox{\Height}{$-1$}}%
%
\polyline(-0.025,1)(0.025,1)%
%
\settowidth{\Width}{$1$}\setlength{\Width}{-1\Width}%
\settoheight{\Height}{$1$}\settodepth{\Depth}{$1$}\setlength{\Height}{\Depth}%
\put( -0.050,  1.025){\hspace*{\Width}\raisebox{\Height}{$1$}}%
%
\polyline(-1.2,0)(1.2,0)%
%
\polyline(0,-1.2)(0,1.2)%
%
\settowidth{\Width}{$x$}\setlength{\Width}{0\Width}%
\settoheight{\Height}{$x$}\settodepth{\Depth}{$x$}\setlength{\Height}{-0.5\Height}\setlength{\Depth}{0.5\Depth}\addtolength{\Height}{\Depth}%
\put(  1.225,  0.000){\hspace*{\Width}\raisebox{\Height}{$x$}}%
%
\settowidth{\Width}{$y$}\setlength{\Width}{-0.5\Width}%
\settoheight{\Height}{$y$}\settodepth{\Depth}{$y$}\setlength{\Height}{\Depth}%
\put(  0.000,  1.225){\hspace*{\Width}\raisebox{\Height}{$y$}}%
%
\settowidth{\Width}{O}\setlength{\Width}{-1\Width}%
\settoheight{\Height}{O}\settodepth{\Depth}{O}\setlength{\Height}{-\Height}%
\put( -0.025, -0.025){\hspace*{\Width}\raisebox{\Height}{O}}%
%
\end{picture}}%}}
\addtext{8}{\ten}{出席確認の代替とします}
\addtext{8}{\ten}{弧度法と正弦曲線の教材}
\addtext{8}{\ten}{弧度法}
\addtext{16}{}{弧の長さ$l$で角度$\theta$を測る}
\addtext{16}{}{弧$l$と半径$r$との比}
\addtext{24}{}{$\theta=\dfrac{l}{r}$}
\addtext[5]{2}{問\monbannoadd}{アプリを動かして,誤差(違い)をできるだけ小さくしてください}
\end{layer}

\addban

\newslide{Cinderellaのインストール}

\vspace*{18mm}

\slidepage
\textinit

\begin{layer}{120}{0}
\addtext{8}{\ten}{Cinderellaは動的幾何の1つ}
\addtext{8}{\ten}{Geogebraなどと同じ}
\addtext{16}{}{汎用のプログラム言語を持つ}
\addtext{2}{問\monbannoadd}{インストールの状況を答えてください}
\end{layer}

\addban
%%%%%%%%%%%%

%%%%%%%%%%%%%%%%%%%%


\newslide{Rのインストール}

\vspace*{18mm}

\slidepage
\textinit

\begin{layer}{120}{0}
\addtext{8}{\ten}{Rは統計ソフトの1つ}
\addtext{16}{}{汎用のプログラム言語を持つ}
\addtext{2}{問\monbannoadd}{インストールの状況を答えてください}
\end{layer}

\addban
%%%%%%%%%%%%

%%%%%%%%%%%%%%%%%%%%


\newslide{KeTCindyのインストール}

\vspace*{18mm}

\slidepage
\textinit

\begin{layer}{120}{0}
\addtext{8}{\ten}{KeTCindyは\TeX 用の図を作成する}
\addtext{8}{\ten}{KeTCindyJSはHTMLを作成する}
\addtext{8}{\ten}{「ketcindy home」からダウンロードできる}
\addtext{8}{\ten}{インストールの仕方}
\addtext{2}{問\monbannoadd}{インストールの状況を答えてください}
\end{layer}

\addban
%%%%%%%%%%%%

%%%%%%%%%%%%%%%%%%%%


\newslide{\TeX のインストール}

\vspace*{18mm}

\slidepage
\textinit

\begin{layer}{120}{0}
\addtext{8}{\ten}{標準ではTeXLiveだが,サイズが大きい}
\addtext{8}{\ten}{そのため,軽量なKeTTeXを提供している}
\addtext{8}{\ten}{インストールの仕方}
\addtext{8}{問\monbannoadd}{インストールの状況を答えてください}
\end{layer}

\addban
\label{pageend}\mbox{}

\end{document}

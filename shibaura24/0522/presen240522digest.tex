%%% title presen240522
\documentclass[landscape,10pt]{ujarticle}
\usepackage{pict2e}
\usepackage{ketpic2e,ketlayer2e}
\special{papersize=\the\paperwidth,\the\paperheight}
\usepackage{ketslide}
\usepackage{amsmath,amssymb}
\usepackage{bm,enumerate}
\usepackage[dvipdfmx]{graphicx}
\usepackage{color}
\usepackage[dvipdfmx,colorlinks=true,linkcolor=blue,filecolor=blue]{hyperref}
\usepackage{emath,emathE,emathMw}

\newcommand{\monthday}{0522}

\definecolor{slidecolora}{cmyk}{0.98,0.13,0,0.43}
\definecolor{slidecolorb}{cmyk}{0.2,0,0,0}
\definecolor{slidecolorc}{cmyk}{0.2,0,0,0}
\definecolor{slidecolord}{cmyk}{0.2,0,0,0}
\definecolor{slidecolore}{cmyk}{0,0,0,0.5}
\definecolor{slidecolorf}{cmyk}{0,0,0,0.5}
\definecolor{slidecolori}{cmyk}{0.98,0.13,0,0.43}
\def\setthin#1{\def\thin{#1}}
\setthin{0}
\newcounter{pagectr}
\setcounter{pagectr}{1}
\newcommand{\slidepage}[1][\monthday-]{%
\setcounter{ketpicctra}{18}%

\begin{layer}{118}{0}
\putnotew{130}{-\theketpicctra.05}{\small#1\thepage/\pageref{pageend}}
\end{layer}

}

\setmargin{25}{145}{15}{100}

\ketslideinit

\pagestyle{empty}

\begin{document}

\begin{layer}{120}{0}
\putnotese{0}{0}{{\Large\bf
\color[cmyk]{1,1,0,0}

\begin{layer}{120}{0}
{\Huge \putnotes{60}{20}{インストール状況}}
\putnotes{60}{50}{高遠節夫}
\end{layer}

}
}
\end{layer}

\def\mainslidetitley{22}
\def\ketcletter{slidecolora}
\def\ketcbox{slidecolorb}
\def\ketdbox{slidecolorc}
\def\ketcframe{slidecolord}
\def\ketcshadow{slidecolore}
\def\ketdshadow{slidecolorf}
\def\slidetitlex{6}
\def\slidetitlesize{1.3}
\def\mketcletter{slidecolori}
\def\mketcbox{yellow}
\def\mketdbox{yellow}
\def\mketcframe{yellow}
\def\mslidetitlex{62}
\def\mslidetitlesize{2}

\color{black}
\Large\bf\boldmath
\addtocounter{page}{-1}

\renewcommand{\slidepage}[1][s]{%
\setcounter{ketpicctra}{18}%
\if#1m \setcounter{ketpicctra}{1}\fi
\hypersetup{linkcolor=black}%
\begin{layer}{118}{0}
\putnotee{115}{-\theketpicctra.05}{\small\monthday-\thepage/\pageref{pageend}}
\end{layer}\hypersetup{linkcolor=blue}
}
%%%%%%%%%%%%

%%%%%%%%%%%%%%%%%%%%

\mainslide{各種ソフトのインストール}


%%%%%%%%%%%%

%%%%%%%%%%%%%%%%%%%%

\newslide{KeTLMSの教材例}

\vspace*{18mm}

\slidepage
\textinit

\begin{layer}{120}{0}
\addtext{8}{\ten}{出席確認の代替とします}
\addtext{8}{\ten}{弧度法と正弦曲線の教材}
\addtext{8}{\ten}{弧度法は弧の長さで角度を測る方法}
\addtext{2}{問\monbannoadd}{アプリを動かして,誤差(違い)をできるだけ小さくしてください}
\end{layer}

\addban
%%%%%%%%%%%%

%%%%%%%%%%%%%%%%%%%%


\newslide{Cinderellaのインストール}

\vspace*{18mm}

\slidepage
\textinit

\begin{layer}{120}{0}
\addtext{8}{\ten}{Cinderellaは動的幾何の1つ}
\addtext{8}{\ten}{Geogebraなどと同じ}
\addtext{16}{}{汎用のプログラム言語を持つ}
\addtext{2}{問\monbannoadd}{インストールの状況を答えてください}
\end{layer}

\addban
%%%%%%%%%%%%

%%%%%%%%%%%%%%%%%%%%


\newslide{Rのインストール}

\vspace*{18mm}

\slidepage
\textinit

\begin{layer}{120}{0}
\addtext{8}{\ten}{Rは統計ソフトの1つ}
\addtext{16}{}{汎用のプログラム言語を持つ}
\addtext{2}{問\monbannoadd}{インストールの状況を答えてください}
\end{layer}

\addban
%%%%%%%%%%%%

%%%%%%%%%%%%%%%%%%%%


\newslide{KeTCindyのインストール}

\vspace*{18mm}

\slidepage
\textinit

\begin{layer}{120}{0}
\addtext{8}{\ten}{KeTCindyは\TeX 用の図を作成する}
\addtext{8}{\ten}{KeTCindyJSはHTMLを作成する}
\addtext{8}{\ten}{「ketcindy home」からダウンロードできる}
\addtext{8}{\ten}{インストールの仕方}
\addtext{2}{問\monbannoadd}{インストールの状況を答えてください}
\end{layer}

\addban
%%%%%%%%%%%%

%%%%%%%%%%%%%%%%%%%%


\newslide{\TeX のインストール}

\vspace*{18mm}

\slidepage
\textinit

\begin{layer}{120}{0}
\addtext{8}{\ten}{標準ではTeXLiveだが,サイズが大きい}
\addtext{8}{\ten}{そのため,軽量なKeTTeXを提供している}
\addtext{8}{\ten}{インストールの仕方}
\addtext{8}{問\monbannoadd}{インストールの状況を答えてください}
\end{layer}

\addban
\label{pageend}\mbox{}

\end{document}

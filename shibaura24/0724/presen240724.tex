%%% title presen240724
\documentclass[landscape,10pt]{jarticle}
\usepackage{pict2e}
\usepackage{ketpic2e,ketlayer2e}
\special{papersize=\the\paperwidth,\the\paperheight}
\usepackage{ketslide}
\usepackage{amsmath,amssymb}
\usepackage{bm,enumerate}
\usepackage[dvipdfmx]{graphicx}
\usepackage{color}
\usepackage[dvipdfmx,colorlinks=true,linkcolor=blue,filecolor=blue]{hyperref}
\usepackage{emath,emathE,emathMw}

\newcommand{\monthday}{0724}

\definecolor{slidecolora}{cmyk}{0.98,0.13,0,0.43}
\definecolor{slidecolorb}{cmyk}{0.2,0,0,0}
\definecolor{slidecolorc}{cmyk}{0.2,0,0,0}
\definecolor{slidecolord}{cmyk}{0.2,0,0,0}
\definecolor{slidecolore}{cmyk}{0,0,0,0.5}
\definecolor{slidecolorf}{cmyk}{0,0,0,0.5}
\definecolor{slidecolori}{cmyk}{0.98,0.13,0,0.43}
\def\setthin#1{\def\thin{#1}}
\setthin{0}
\newcounter{pagectr}
\setcounter{pagectr}{1}
\newcommand{\slidepage}[1][\monthday-]{%
\setcounter{ketpicctra}{18}%

\begin{layer}{118}{0}
\putnotew{130}{-\theketpicctra.05}{\small#1\thepage/\pageref{pageend}}
\end{layer}

}

\setmargin{25}{145}{15}{100}

\ketslideinit

\pagestyle{empty}

\begin{document}

\begin{layer}{120}{0}
\putnotese{0}{0}{{\Large\bf
\color[cmyk]{1,1,0,0}

\begin{layer}{120}{0}
{\Huge \putnotes{60}{20}{KeTCindyの利用}}
\putnotes{60}{50}{高遠節夫}
\putnotes{60}{70}{2024.07.10}
\end{layer}

}
}
\end{layer}

\def\mainslidetitley{22}
\def\ketcletter{slidecolora}
\def\ketcbox{slidecolorb}
\def\ketdbox{slidecolorc}
\def\ketcframe{slidecolord}
\def\ketcshadow{slidecolore}
\def\ketdshadow{slidecolorf}
\def\slidetitlex{6}
\def\slidetitlesize{1.3}
\def\mketcletter{slidecolori}
\def\mketcbox{yellow}
\def\mketdbox{yellow}
\def\mketcframe{yellow}
\def\mslidetitlex{62}
\def\mslidetitlesize{2}

\color{black}
\Large\bf\boldmath
\addtocounter{page}{-1}

\renewcommand{\eda}[2][\theedactr]{%
\Ltab{\theedawidth mm}{[#1]\ #2}%
\addtocounter{edactr}{1}%
}
\renewcommand{\slidepage}[1][s]{%
\setcounter{ketpicctra}{18}%
\if#1m \setcounter{ketpicctra}{1}\fi
\hypersetup{linkcolor=black}%
\begin{layer}{118}{0}
\putnotee{115}{-\theketpicctra.05}{\small\monthday-\thepage/\pageref{pageend}}
\end{layer}\hypersetup{linkcolor=blue}
}
\newcommand{\bs}{$\backslash$}
%%%%%%%%%%%%

%%%%%%%%%%%%%%%%%%%%

\mainslide{補足}


%%%%%%%%%%%%

%%%%%%%%%%%%%%%%%%%%

\newslide{\ketcindy JSによるアニメーション}

\vspace*{18mm}

\slidepage
\textinit

\begin{layer}{120}{0}
\addtext{8}{\ten}{06animationjs.cdyを用いる}
\end{layer}

%%%%%%%%%%%%

%%%%%%%%%%%%%%%%%%%%


\newslide{\bs inputと\bs includegraphics}

\vspace*{18mm}

\slidepage
\textinit

\begin{layer}{120}{0}
\addtext{8}{\ten}{$\backslash$input}
\addtext{18}{}{描画コードファイルの読み込み}
\addtext{8}{\ten}{$\backslash$includegraphics}
\addtext{18}{}{png,pdfなどの読み込み}
\addtext{18}{}{\ketcindy で描画領域を制限したPDF作成}
\addtext{18}{}{bounding boxのエラーが出た時の対応}
\end{layer}

%%%%%%%%%%%%

%%%%%%%%%%%%%%%%%%%%


\newslide{layer環境}

\vspace*{18mm}

\slidepage
\textinit

\begin{layer}{120}{0}
\addtext{8}{\ten}{図表を思った位置に配置する}
\addtext{12}{}{\bs begin\{layer\}\{140\}\{70\}}
\addtext{30}{}{単位はmm}
\addtext{30}{}{最後の引数を0とするとグリッドが消える}
\addtext{12}{}{\bs putnotese\{50\}\{30\}\{図\}}
\addtext{30}{}{s,n,及びs,nとe,wの組,及びcが可能}
\addtext{12}{}{\bs end\{layer\}}
\end{layer}

%%%%%%%%%%%%

%%%%%%%%%%%%%%%%%%%%


\newslide{自由課題}

\vspace*{18mm}

\slidepage
\textinit

\begin{layer}{120}{0}
\addtext{8}{問\monbannoadd}{Answer the question}
\end{layer}

%%%%%%%%%%%%

%%%%%%%%%%%%%%%%%%%%


\sameslide

\vspace*{18mm}

\textinit

\begin{layer}{120}{0}
\addtext{8}{問\monbannoadd}{Answer the question}
\addban
\addtext{8}{\ten}{幾何点をスクリプトでとる方法}
\addtext{16}{}{Putpoint("A",[1,2],A.xy)}
\addtext{8}{問\monbannoadd}{\ketcindy JSか\ketcindy で1つの作品を作れ}
\addban
\end{layer}


\newslide{授業後アンケート}

\vspace*{18mm}

\slidepage
\textinit

\begin{layer}{120}{0}
\addtext{8}{問\monbannoadd}{今年度の授業について}
\addtext{16}{[1]}{自分の到達度}
\addtext{16}{[2]}{授業についての感想}
\addtext{16}{[3]}{授業についての要望など}
\end{layer}

\addban
\label{pageend}\mbox{}

\end{document}

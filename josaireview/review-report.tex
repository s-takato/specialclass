\documentclass[11pt]{jarticle}
\usepackage{amsmath,amssymb,graphicx,color}
\usepackage{url}


\topmargin =-1cm 
\oddsidemargin =0cm
\textheight=23.5cm
\textwidth=15.7cm

\begin{document}

\begin{center}
{\Large\bf 査察報告書}
\end{center}

\vspace{5mm}
\noindent
{\bf 論文題目:}\hspace{2zw}%
円錐曲線と反射光線に関する連動型 GeoGebra 教材とその効果

\noindent
{\bf 論文著者:}\hspace{2zw}%
日本大学大学院理工学研究科 室井龍二, 川口桃花, 鷲尾勇介,\\
 \phantom{{\bf 論文著者:}\hspace{2zw}}日本大学理工学部 鷲尾夕紀子,谷部貴一,鈴木潔光,利根川聡,古津博俊,平田典子

\noindent
{\bf 査読者:}\hspace{3zw}%
高遠\ 節夫(KeTCindyセンター)

\noindent
{\bf 査読年月日:}\hspace{1zw}%
2023年8月31日

\vspace*{10mm}
\begin{center}
{\large\bf 所\hspace{2zw}見}
\end{center}

\begin{enumerate}
\item%1.
{\bf 評価できる点}
\begin{enumerate}
\item[(1)]%(1)
2次曲線を教材化しようとする試み.
\end{enumerate}

\item%2
{\bf このままでは掲載できないとする理由}
\begin{enumerate}
\item[(1)]%(1)
授業の対象者を明確にして,授業デザインを綿密に考えてほしい.
\item[(2)]%(2)
対象者が2次曲線初習の場合は,離心率の定義(1)の箇所でまず動的幾何を用いて,
(1)の式の意味を理解してもらう必要があり,そのための動的教材も有用である.

\item[(3)]%(3)
対象者が既習の場合は,円錐切断で2次曲線が現れることを動的幾何の3Dで示すことは
意味があり得るが,3Dとリンクさせて放物線で1点に集まることを説明するのは無理があり,
むしろ2Dのアニメーションで点が反射するようすを見せるほうがいいと思われる.
\item[(4)]%(4)
放物線だけでなく,楕円や双曲線で1つの焦点からの光の反射も重要である.
しかし,(6)式のようには導出されないので,この方法では難しいかもしれない
(計算がかなり複雑になる).Geogebraでどの程度計算できるかはわからないが,
場合によっては数式処理の利用が必要かつ有用と思われる.
\item[(5)]%(5)
事前事後のクイズは,あまりにも単純で,これだけでは効果の測定は難しい.
数学の問題としては,この後「なぜか」「他の2次曲線ではどうなるか」などを
考えさせて,深みを増す工夫が必要である.
\item[(6)]%(6)
QRコードをどのタイミングでどう利用しているかが不明確である.「2次曲線について」の説明を
印刷して配付する場合は必要となるであろうが,GoogleFormなどで作成して,その最後にQRコードを付けているとすれば,
説明部分を学生が独力で理解するのは難しいため,教員がファイルに沿った説明をし.そのファイルまたは関連ファイルを送信することになるので,ファイルにURLを書き込んだ方が直接的であり,スマホでもPCでも使えるはずである.

\end{enumerate}

\item%3
{\bf 修正を要する点}
%\begin{enumerate}
% \item[(1)]%(1)
%\end{enumerate}


\item%4
{\bf 提案・コメント}
%********************************

以下のURLにHTMLの例を作ったので参考にしてください

{\small
\url{https://s-takato.github.io/specialclass/josaireview/quadrajson.html}
}

{\small
\url{https://s-takato.github.io/specialclass/josaireview/parabolajson.html}
}


%********************************
\end{enumerate}

\end{document}